\section{Diskussion}
Bei der Bestimmung der Totzeit mithilfe des Oszilloskops ist die Zeit relativ schlecht abzulesen.
Leider bestand nicht die Möglichkeit ein Standbild von dieser Messung zu machen, da das Oszilloskop nicht die nötige
Ausstattung hatte. Ein Literaturwert zur Abgleichung unserer Aufnahme gab es nicht.
Einzige Möglichkeit um die Totzeit vergleichen zu können war die Totzeit von der Zwei-Quellen-Methode.
Die Abweichung da ist bei $\approx 100 \, \%$.
Bei der anderen Messmethode fällt auf, dass die Totzeit im negativen Bereich liegt und das hätte nicht sein dürfen,
da die Bedingung $N_{1+2} < N_1 + N_2$ nicht erfüllt worden ist.

Erfreulicherweiße sind die Anzahl der Teilchen in der richtigen Größenordnung zu finden.
