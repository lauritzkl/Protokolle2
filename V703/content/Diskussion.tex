\section{Diskussion}
Für die Plateausteigung ergibt sich $\SI{1.2(4)}{\%}$. Die Plateausteigung sollte
unterhalb von \SI{2}{\%} liegen was bei diesem Zählrohr zutrifft \cite{2}.

Bei der Bestimmung der Totzeit mithilfe des Oszilloskops ist die Zeit relativ schlecht abzulesen.
Leider bestand nicht die Möglichkeit ein Standbild von dieser Messung zu machen, da das Oszilloskop nicht die nötige
Ausstattung hatte. Die bestimmte Totzeit ist $T_\text{tot} = \SI{0.20(1)}{\milli\second}$
und die Erholungszeit $T_\text{re} = \SI{0.67(19)}{\milli\second}$.
Ein Literaturwert zur Abgleichung dieser Aufnahme gab es nicht.
Bei der anderen Messmethode fällt auf, dass die Totzeit im negativen Bereich liegt und das hätte nicht sein dürfen,
da die Bedingung $N_{1+2} < N_1 + N_2$ nicht erfüllt worden ist.

Erfreulicherweiße sind die Anzahl der Teilchen in der richtigen Größenordnung zu finden.
