\section{Diskussion}
Es werden nochmal die Fehlerabweichungen vom A-Scan und vom B-Scan
gegenübergestellt und in der Tabelle (\ref{tab:8}) dargestellt.
\begin{table}[H]
  \centering
  \caption{Fehlerabweichungen.}
  \label{tab:8}
  \begin{tabular}{c c c c}
    \toprule
    %\multicolumn{1}{c}{A-Scan} & \multicolumn{1}{c}{B-Scan}\\
    %\cmidrule(lr){3}\cmidrule(lr){4}
    $\text{Lochnummer}$&
    $\diameter_{\text{Theorie}} \,/\, \si[per-mode=fraction]{\milli\meter}$&
    $\text{Fehlerabweichung A-Scan}  \,/\, \%$&
    $\text{Fehlerabweichung B-Scan} \,/\, \%$\\
    \midrule
    1   & 1,25 & 60  & 100  \\
    2   & 1,25 & 60  & 86,6 \\
    3   & 5,30 & 6   & 0    \\
    4   & 4,30 & 7   & 4,7  \\
    5   & 3,40 & 12  & 11,8 \\
    6   & 2,45 & 18  & 18,4 \\
    7   & 2,45 & 18  & 14,3 \\
    8   & 2,45 & 18  & 6,1  \\
    9   & 2,45 & 51  & 6,1  \\
    10  & 2,45 & 18  & 39,6 \\
    11  & 9,35 &  4  & 0,5  \\
  \bottomrule
  \end{tabular}
\end{table}
In einigen Fehlstellen sind die Abweichungen der jeweiligen Scans identisch. In den
anderen sind die Abweichungen groß. Eine Aussage über welchen Scan besser ist, ist
hier nicht möglich. \\
Zur Auflösung mit den unterschiedlichen Sonden hebt die Abbildung (\ref{abb:4}) am
besten hervor. Deutlich zu erkennen sind die zwei Peaks nebeneinander. Dies
sagt aus, dass aufgrund der hohen Frequenz die Wellenlänge sehr klein ist und
die eng benachbarten Fehlstellen besser zu differenzieren ist.
Zum Herzmodell kann keine richtige Aussage über
die Messung gemacht werden, da wir keine Vergleichswerte
haben.
