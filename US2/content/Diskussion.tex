\section{Diskussion}
Es werden nochmal die Abweichungen vom A-Scan und vom B-Scan
gegenübergestellt und in der Tabelle (\ref{tab:8}) dargestellt.
\begin{table}[H]
  \centering
  \caption{Abweichungen.}
  \label{tab:8}
  \begin{tabular}{c c c c}
    \toprule
    %\multicolumn{1}{c}{A-Scan} & \multicolumn{1}{c}{B-Scan}\\
    %\cmidrule(lr){3}\cmidrule(lr){4}
    $\text{Lochnummer}$&
    $\diameter_{\text{Schieb}} \,/\, \si[per-mode=fraction]{\milli\meter}$&
    $\text{Abweichung A-Scan}  \,/\, \%$&
    $\text{Abweichung B-Scan} \,/\, \%$\\
    \midrule
    1   & 1,25 & 60  & 100  \\
    2   & 1,25 & 60  & 86,6 \\
    3   & 5,30 & 6   & 0    \\
    4   & 4,30 & 7   & 4,7  \\
    5   & 3,40 & 12  & 11,8 \\
    6   & 2,45 & 18  & 18,4 \\
    7   & 2,45 & 18  & 14,3 \\
    8   & 2,45 & 18  & 6,1  \\
    9   & 2,45 & 51  & 6,1  \\
    10  & 2,45 & 18  & 39,6 \\
    11  & 9,35 &  4  & 0,5  \\
  \bottomrule
  \end{tabular}
\end{table}
In einigen Fehlstellen sind die Abweichungen der jeweiligen Scans identisch. Bei den
Fehlstellen 1 und 2 sind die Abweichungen bei beiden Verfahren sehr groß. Das liegt
daran, dass das Auflösungsvermögen der Sonde zu schlecht war.
Eine Aussage darüber welcher Scan besser ist, ist nicht möglich.

Bei der Untersuchung des Auflösungsvermögens von unterschiedlichen Sonden fällt auf,
dass die Auflösung besser wird je höher die Frequenz der Sonde ist. Dies
sagt aus, dass aufgrund der hohen Frequenz die Wellenlänge sehr klein ist und
die eng benachbarten Fehlstellen besser zu differenzieren sind.

Zum Herzmodell kann keine richtige Aussage über
die Messung gemacht werden, da keine Vergleichswerte vorliegen.
