\section{Theorie}

In diesem Versuch geht es um Ultraschalltechnik. Es wird von Ultraschall gesprochen
bei Frequenzen von ungefähr \SI{20}{\kilo\hertz} bis \SI{1}{\giga\hertz}. Schall ist
eine longitudinale Welle, die sich durch Druckschwankungen fortbewegt. Diese Wellen
können mit folgender Gleichung beschrieben werden:

\begin{equation*}
  p(x,t) = p_0 + v_0 Z \cos(\omega t - k x).
\end{equation*}

Bei $Z = c \cdot \rho$ handelt es sich um die akustische Impedanz, die von der
Dichte $\rho$ und der Schallgeschwindigkeit $c$ des Materials abhängt, durch welches
sich die Schallwelle bewegt. Für die Bestimmung der Schallgeschwindigkeit kommt es
dadrauf an, ob es sich um ein Gas, eine Flüssigkeit oder um einen Festkörper handelt.
Die Schallgeschwindigkeit in einer Flüssigkeit lässt sich mit folgender Formel bestimmen:

\begin{equation*}
  c_{Fl} = \sqrt{\frac{1}{\kappa \cdot \rho}}
\end{equation*}

wobei $\kappa$ die Kompressibilität der Flüssigkeit ist. Die Formel für einen Festkörper
unterscheidet sich von der für Flüssigkeiten aufgrund der Schubspannung. Außerdem kommt es
auch zu Transversalwellen in Festkörpern, welche eine andere Schallgeschwindigkeit haben.
Es ergibt sich für die Berechnung der Schallgeschwindigkeit in einem Festkörper die
Gleichung:

\begin{equation*}
  c_{Fe} = \sqrt{\frac{E}{\rho}}.
\end{equation*}

Dabei ist E das Elastizitätsmodul.

Die Intensität einer Schallwelle nimmt exponentiell ab, dabei kommt es auf den
Absorptionskoeffizienten $\alpha$ der Schallamplitude von dem vorliegenden Material an.

\begin{equation*}
  I (x) = I_0 \cdot \exp(-\alpha x)
\end{equation*}

Außerdem wird ein Teil einer Schallwelle bei einer Grenzfläche von zwei Materialien
reflektiert und ein Teil transmittiert. Für die Berechnung des Reflexionskoeffizienten
werden die akustischen Impedanzen $Z$ der Materialien benötigt.

\begin{equation*}
  R = \left( \frac{Z_1 - Z_2}{Z_1 + Z_2} \right)^2
\end{equation*}

Für die Erzeugung von Ultraschallwellen wird meistens der \enquote{piezo-elektrische Effekt}
benutzt. Dieser Effekt besagt, dass ein Piezokristall, zum Beispiel Quarze, in einem
elektrischen Wechselfeld anfangen zu schwingen und somit Schallwellen abstrahlen.
Ist die Anregungsfrequenz gleich der Eigenfrequenz des Kristalls kommt es zu Resonanz
und somit zu hohen Schallenergiedichten.
Der Piezokristall kann außerdem auch als Empfänger für Ultraschallwellen verwendet
werden, da er auch durch diese Wellen zum schwingen angeregt wird.

Bei der Ultraschalltechnik werden zwei Verfahren angewendet.

\begin{enumerate}
  \item \underline{Durchschallungs-Verfahren:} Hierbei wird auf einer Seite des Probestücks
  ein Schallimpuls ausgesendet und auf der anderen Seite mit einem Empfänger aufgefangen.
  Falls sich in der Probe andere Materialien befinden, wird das durch eine abgeschwächte
  Intensität gemessen. Es kann allerdings nicht festgestellt werden wo sich die anderen
  Materialien befinden.
  \item \underline{Impuls-Echo-Verfahren:} Bei diesem Verfahren ist der Sender auch der Empfänger
  und es wird die Intensität der reflektierten Schallwellen gemessen. Außerdem kann
  durch Laufzeitmessungen der Ort von Verunreinigungen in der Probe bestimmt werden.
  Dafür muss allerdings die Schallgeschwindigkeit des Materials bekannt sein für die
  folgende Gleichung.
\end{enumerate}

\begin{equation}
  s = \frac{1}{2} c t
  \label{eq:1}
\end{equation}

In der Medizin wird überwiegend das Impuls-Echo-Verfahren verwendet. Dabei können die
Laufzeitmessungen auf verschiedene Arten ausgewertet werden.

\begin{itemize}
  \item Amplituden-Scan: Die Echoamplituden werden als Funktion der Laufzeit dargestellt.
  Dieser Scan ist ein eindimensionales Verfahren.
  \item Brightness-Scan: Dabei wird die Sonde bewegt und die Echoamplituden werden in
  Helligkeitsabstufungen dargestellt. Es ergibt sich ein zweidimensionales Bild.
  \item Time-Motion-Scan: Dabei wird eine zeitliche Bildfolge dargestellt, welche
  durch schnelle Abtastung entsteht.
\end{itemize}
