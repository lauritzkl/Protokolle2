\section{Auswertung}
\subsection{Vorbereitung}
Als Vorbereitung wird zunächst die Schallgeschwindigkeiten von einigen Materialien bestimmt. Sie sind in
der Tabelle (\ref{tab:1}) dargestellt.
\begin{table}[H]
  \centering
  \caption{Literaturwerte für Schallgeschwindigkeiten \cite{2} \cite{3}.}
  \label{tab:1}
  \begin{tabular}{c c}
    \toprule
     & $c \, / \, \si[per-mode=fraction]{\meter\per\second}$\\
     \midrule
     Luft \SI{0}{\celsius}          & 331  \\
     dest. Wasser \SI{20}{\celsius} & 1485 \\
     Acryl                          & 2730 \\
     \bottomrule
  \end{tabular}
\end{table}

\subsection{Untersuchung eines Acrylblocks mit dem A-Scan}
Zunächst werden die Abmessungen vom Acrylblocks mithilfe einer Schieblehre gemessen und in
der Tabelle (\ref{tab:2}) dargestellt.

\begin{table}[H]
  \centering
  \caption{Abmessung des Acrylblocks.}
  \label{tab:2}
  \begin{tabular}{c c c}
    \toprule
     $\text{Länge} \,/\, \si[per-mode=fraction]{\milli\meter}$ &$\text{Höhe} \, /\, \si[per-mode=fraction]{\milli\meter}$ &$\text{Breite} \, /\, \si[per-mode=fraction]{\milli\meter}$\\
     \midrule
     150 & 80.01 & 60 \\
     \bottomrule
  \end{tabular}
\end{table}

Anschließend werden die Tiefen der Fehlstellen mit dem A-Scan bestimmt. Die dazugehörige Schallgeschwindigkeit
von Acryl wird von der Tablle{\ref{tab:1}} entnommen und mit der Formel (\ref{eq:1}) nach der Strecke umgeformt.
Um eine genau Lokalisation der Fehlstelle zu bekommen wird der Block zusätzlich von der andere Seite durchschallt.
Die Messaufnahmen sind in der Tabelle (\ref{tab:3}) dargestellt.

\begin{table}[H]
  \centering
  \caption{Messergebnisse vom A-Scan.}
  \label{tab:3}
  \begin{tabular}{c c c c}
    \toprule
    \multicolumn{2}{c}{Oben} & \multicolumn{2}{c}{Unten} \\
    \cmidrule(lr){1-2}\cmidrule(lr){3-4}
    $\text{Lochnummer}$ & $\text{Tiefe} \,/\, \si[per-mode=fraction]{\milli\meter}$ & $\text{Lochnummer}$ & $\text{Tiefe} \,/\, \si[per-mode=fraction]{\milli\meter}$\\
    \midrule
    11  & 56 & 11 & 15
    10  & 7  & 10 & 71
    9   & 16 & 9 & 63
    8   & 23 & 8 & 55
    7   & 31 & 7 & 47
    6   & 39 & 6 & 39
    5   & 47 & 5 & 30
    4   & 54 & 4 & 22
    3   & 62 & 3 & 13
    2   & 18 & 2 & 64
    1   & 22 & 1 & 60
  \bottomrule
  \end{tabular}
\end{table}

Der Durchmesser der jeweiligen Fehlstellen wird mit
\begin{itemize}
  \item $\text{Durchmesser} = |\text{Höhe} - (Messwert_{Oben} + Messwert_{Unten})|$
\end{itemize}
berechnet und in der Tabelle (\ref{tab:4}) mit den tatsächlichen Abmessungen gegenübergestellt.
Für die Fehlerabweichung wird diese Formel verwendet.
\begin{itemize}
  \item $\text{Fehlerabweichung} = |\frac{\varnothing_{Theo} \cdot \varnothing_{Mess}}{\varnothing_{Theo}}| \cdot 100 \, \%$
\end{itemize}

%\begin{table}[H]
%  \centering
%  \caption{Durchmesser der jeweiligen Fehlstellen.}
%  \label{tab:4}
%  \begin{tabular}{c c c c c}
%    \toprule
%    & & & & \multicolumn{2}{c}{Genau Lokalisation von Oben} \\
%    \cmidrule(lr){5}
%     $\text{Lochnummer}$ &
%     $\varnothing_{\text{Theorie}}} \,/\, \si[per-mode=fraction]{\milli\meter}$ &
%     $\varnothing_{\text{Messung}}} \,/\, \si[per-mode=fraction]{\milli\meter}$ &
%     $\text{Fehlerabweichung} \,/\, \%$ &
%     $\si[per-mode=fraction]{\milli\meter}$ \\
%     \midrule
%     1 & 1,25 & 2 &
%     2 &
%     3 &
%     4 &
%     5 &
%     6 &
%     7 &
%     8 &
%     9 &
%     10 &
%     11 &
%     \bottomrule
%  \end{tabular}
%\end{table}
