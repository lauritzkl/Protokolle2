\section{Diskussion}
Für die $\varphi$-Messung ergab sich der Winkel $\varphi = (60,16\pm0,03)\textdegree$.
In der Theorie sollte der Winkel bei $60 \textdegree$ liegen. Dies ist eine Abweichung
von $0,27 \, \%$ und liegt im Toleranzbereich.\newline

Für die $\eta$-Messung und der \textbf{Abbesche Zahl} sind keine Vergleichswerte
für uns bekannt. Daher kann keine Aussage über die Richtigkeit der Messung gemacht werden.\newline

Zur Dispersionskurve wurde der Fall $\lambda >> \lambda_1$ genommen, da die
Methode der kleinsten Quadrate einen Wert von $s^2_{n} = 1,928 \cdot 10^{-5}$ ergab. Dies
ist gegenüber dem anderen Wert $s^2_{n'} = 0,0003$ kleiner.\newline

Zur Auflösungsvermögen kann keine Aussage gemacht, da keine Literaturwerte gegeben sind.

Zur Absorptionstelle $\lambda_1 = \SI{174,9}{\nano\meter}$ kann die Aussage gemacht werden, dass die
nächste Stelle im Ultravioletten-Bereich\cite{2} liegt.
