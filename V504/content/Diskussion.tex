\section{Diskussion}

Bei der graphischen Darstellung der Kennlinien lässt sich gut erkennen, dass
der Sättigungsstrom mit der Temperatur zunimmt und nicht von der Spannung abhängt.
Außerdem wird an der graphischen Abbildung der Messwerte gut sichtbar, dass sie
Aussehen wie die Kennlinie, die in Abbildung \ref{abb:3} gezeigt ist.

Bei der Überprüfung des Langmuir-Schottkyschen Raumladungsgesetz passt der Fit
ziemlich gut auf die Messwerte und der Fehler der bestimmten Parameter, die
$\Delta a = \num{0.21e-3}$ und $\Delta b = \num{0.016}$ betragen, sind auch mindestens eine
Größenordnung kleiner als die Werte.
Der bestimmte Exponent für das Raumladungsgesetz ist $b = \num{1.425(16)}$. Der
Theoretische Exponent von diesem Gesetz ist $b_\text{theo} = 1,5$ was einer Abweichung
von $5 \%$. Bei der Ausgleichsrechnung wurde ein geringeres Spannungsintervall
verwendet als angenommen, da das Ende des ursprünglichen Intervalls bereits in das
Sättigungsstromgebiet übergeht.

Bei der Untersuchung des Anlaufstromgebietes ist bei der Messung bereits aufgefallen,
dass die Messwerte nicht sehr genau seien können, da das Nanoamperemeter bei der
Messung stark geschwankt ist. Bei den Korrigierten Werten fällt außerdem
auch auf, dass zwei gemessene Spannungen negativ wurden, was auch auf eine
ungenaue Messung hinweist. Das äußert sich auch in der bestimmten Temperatur.
Die Temperatur ist $T = \SI{3440(170)}{\kelvin}$ und diese Temperatur liegt nicht in
den Größenordnung von den im nächsten Teil bestimmten Temperaturen. Außerdem ist der
Schmelzpunkt von Wolfram $ T_s = \SI{3695}{\kelvin}$ \cite{2}, was sehr nah an dem
bestimmten Wert ist. Deshalb ist der Wert nicht sinnvoll.

Bei dem letzten Versuchsteil wird die Austrittsarbeit von Wolfram bestimmt, die
sich ergeben hat zu $ \overline{(e_0\Phi)} = \SI{5.095(60)}{\eV}$. Der Theoriewert der Austrittsarbeit
lautet $ (e_0 \Phi)_\text{theo} = \SI{4.54}{\eV}$ \cite{3}. Das ergibt eine Abweichung von $12,22 \%$.
Auch diese Abweichung ist relativ hoch, was dadrauf schließen lässt, dass die Messungen
insgesamt nicht sehr genau waren.
