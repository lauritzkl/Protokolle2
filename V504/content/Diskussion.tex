\section{Diskussion}

Bei der graphischen Darstellung der Kennlinien lässt sich gut erkennen, dass
der Sättigungsstrom mit der Temperatur zunimmt und nicht von der Spannung abhängt.

Der bestimmte Exponent für das Raumladungsgesetz ist $b = \num{1.165(3)}$. Der
Theoretische Exponent von diesem Gesetz ist $b_\text{theo} = 1,5$ was einer Abweichung
von $22,33 \%$. Diese Abweichung ist ziemlich groß. Das könnte dadran liegen, dass
nicht das gesamte Raumladungsgebiet gemessen wurde, da der Bereich nicht exakt abgegrenzt
werden konnte.

Bei der Untersuchung des Anlaufstromgebietes ist bei der Messung bereits aufgefallen,
dass die Messwerte nicht sehr genau seien können, da das Nanoamperemeter bei der
Messung stark geschwankt ist. Das äußert sich auch in der bestimmten Temperatur.
Die Temperatur ist $T = \SI{4340(250)}{\kelvin}$ und diese Temperatur liegt nicht in
den Größenordnung von den im nächsten Teil bestimmten Temperaturen. Außerdem ist der
Schmelzpunkt von Wolfram $ T_s = \SI{3695}{\kelvin}$ \cite{2}.

Bei dem letzten Versuchsteil wird die Austrittsarbeit von Wolfram bestimmt, die
sich ergeben hat zu $ \bar{\Phi} = \SI{5.095(60)}{\eV}$. Der Theoriewert der Austrittsarbeit
lautet $ \Phi_\text{theo} = \SI{4.54}{\eV}$ \cite{3}. Das ergibt eine Abweichung von $12,22 \%$.
Auch diese Abweichung ist relativ hoch, was dadrauf schließen lässt, dass die Messungen
insgesamt nicht sehr genau waren.
