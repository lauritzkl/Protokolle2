\section{Auswertung}

Zunächst werden die Kennlinien bei verschiedenen Heizleistungen bestimmt. Da
die Temperatur nicht gemessen werden kann wird die Stromstärke notiert. Die
Messwerte der einzelnen Messungen sind in Tabelle \ref{tab:1} dargestellt.

\begin{table}[H]
  \centering
  \caption{Messwerte der Messungen für die Kennlinien.}
  \label{tab:1}
  \begin{tabular}{c c c c c c c c c c}
\toprule
\multicolumn{2}{c}{$I=\SI{1.8}{\ampere}$} & \multicolumn{2}{c}{$I=\SI{1.9}{\ampere}$} & \multicolumn{2}{c}{$I=\SI{2.0}{\ampere}$} & \multicolumn{2}{c}{$I=\SI{2.2}{\ampere}$} & \multicolumn{2}{c}{$I=\SI{2.4}{\ampere}$}\\
\cmidrule(lr){1-2}\cmidrule(lr){3-4}\cmidrule(lr){5-6}\cmidrule(lr){7-8}\cmidrule(lr){9-10}
$U \, / \, V$ & $I \, / \, mA$ & $U \, / \, V$ & $I \, / \, mA$ &$U \, / \, V$ & $I \, / \, mA$ & $U \, / \, V$ & $I \, / \, mA$ & $U \, / \, V$ & $I \, / \, mA$ \\
\midrule
0   & 0,000 & 0   & 0,000 & 0   & 0,000 & 0   & 0,000 & 0   & 0,000 \\
5   & 0,001 & 5   & 0,000 & 5   & 0,000 & 5   & 0,013 & 2   & 0,004 \\
10  & 0,008 & 10  & 0,000 & 10  & 0,004 & 10  & 0,053 & 4   & 0,016 \\
15  & 0,013 & 15  & 0,003 & 15  & 0,023 & 15  & 0,108 & 6   & 0,033 \\
20  & 0,014 & 20  & 0,011 & 20  & 0,044 & 20  & 0,166 & 8   & 0,067 \\
25  & 0,015 & 25  & 0,023 & 25  & 0,066 & 25  & 0,236 & 10  & 0,088 \\
30  & 0,016 & 30  & 0,031 & 30  & 0,080 & 30  & 0,285 & 12  & 0,122 \\
35  & 0,016 & 35  & 0,035 & 35  & 0,086 & 35  & 0,331 & 14  & 0,150 \\
40  & 0,016 & 40  & 0,037 & 40  & 0,089 & 40  & 0,358 & 16  & 0,180 \\
45  & 0,017 & 45  & 0,038 & 45  & 0,091 & 45  & 0,376 & 18  & 0,217 \\
50  & 0,017 & 50  & 0,039 & 50  & 0,092 & 50  & 0,387 & 20  & 0,250 \\
55  & 0,017 & 55  & 0,039 & 55  & 0,093 & 55  & 0,394 & 22  & 0,290 \\
60  & 0,017 & 60  & 0,040 & 60  & 0,095 & 60  & 0,400 & 24  & 0,332 \\
70  & 0,017 & 70  & 0,040 & 70  & 0,095 & 70  & 0,407 & 26  & 0,366 \\
80  & 0,017 & 80  & 0,041 & 80  & 0,096 & 80  & 0,411 & 28  & 0,407 \\
90  & 0,017 & 90  & 0,041 & 90  & 0,096 & 90  & 0,414 & 30  & 0,453 \\
100 & 0,017 & 100 & 0,041 & 100 & 0,097 & 100 & 0,417 & 32  & 0,499 \\
110 & 0,018 & 110 & 0,041 & 110 & 0,097 & 110 & 0,419 & 34  & 0,545 \\
120 & 0,018 & 120 & 0,042 & 120 & 0,098 & 120 & 0,422 & 36  & 0,585 \\
130 & 0,018 & 130 & 0,042 & 130 & 0,098 & 130 & 0,423 & 38  & 0,625 \\
140 & 0,018 & 140 & 0,042 & 140 & 0,099 & 140 & 0,425 & 40  & 0,660 \\
150 & 0,018 & 150 & 0,042 & 150 & 0,099 & 150 & 0,427 & 42  & 0,699 \\
160 & 0,018 & 160 & 0,043 & 160 & 0,099 & 160 & 0,428 & 44  & 0,735 \\
170 & 0,018 & 170 & 0,043 & 170 & 0,100 & 170 & 0,430 & 46  & 0,768 \\
180 & 0,018 & 180 & 0,043 & 180 & 0,100 & 180 & 0,431 & 48  & 0,808 \\
190 & 0,018 & 190 & 0,043 & 190 & 0,100 & 190 & 0,432 & 50  & 0,852 \\
200 & 0,018 & 200 & 0,043 & 200 & 0,100 & 200 & 0,433 & 52  & 0,891 \\
210 & 0,018 & 210 & 0,043 & 210 & 0,101 & 210 & 0,434 & 54  & 0,921 \\
220 & 0,018 & 220 & 0,043 & 220 & 0,101 & 220 & 0,435 & 56  & 0,959 \\
230 & 0,018 & 230 & 0,043 & 230 & 0,101 & 230 & 0,436 & 58  & 0,983 \\
240 & 0,018 & 240 & 0,044 & 240 & 0,101 & 240 & 0,437 & 60  & 1,013 \\
250 & 0,018 & 250 & 0,044 & 250 & 0,102 & 250 & 0,438 & 70  & 1,165 \\
 -  &   -   &  -  &   -   &  -  &   -   &  -  &   -   & 80  & 1,274 \\
 -  &   -   &  -  &   -   &  -  &   -   &  -  &   -   & 90  & 1,341 \\
 -  &   -   &  -  &   -   &  -  &   -   &  -  &   -   & 100 & 1,406 \\
 -  &   -   &  -  &   -   &  -  &   -   &  -  &   -   & 110 & 1,438 \\
 -  &   -   &  -  &   -   &  -  &   -   &  -  &   -   & 120 & 1,463 \\
 -  &   -   &  -  &   -   &  -  &   -   &  -  &   -   & 130 & 1,481 \\
 -  &   -   &  -  &   -   &  -  &   -   &  -  &   -   & 140 & 1,491 \\
 -  &   -   &  -  &   -   &  -  &   -   &  -  &   -   & 150 & 1,501 \\
 -  &   -   &  -  &   -   &  -  &   -   &  -  &   -   & 160 & 1,507 \\
 -  &   -   &  -  &   -   &  -  &   -   &  -  &   -   & 170 & 1,514 \\
 -  &   -   &  -  &   -   &  -  &   -   &  -  &   -   & 180 & 1,520 \\
 -  &   -   &  -  &   -   &  -  &   -   &  -  &   -   & 190 & 1,524 \\
 -  &   -   &  -  &   -   &  -  &   -   &  -  &   -   & 200 & 1,530 \\
 -  &   -   &  -  &   -   &  -  &   -   &  -  &   -   & 210 & 1,534 \\
 -  &   -   &  -  &   -   &  -  &   -   &  -  &   -   & 220 & 1,538 \\
 -  &   -   &  -  &   -   &  -  &   -   &  -  &   -   & 230 & 1,542 \\
 -  &   -   &  -  &   -   &  -  &   -   &  -  &   -   & 240 & 1,545 \\
 -  &   -   &  -  &   -   &  -  &   -   &  -  &   -   & 250 & 1,550 \\
\bottomrule
  \end{tabular}
\end{table}

Die Messwerte sind in Abbildung \ref{abb:6} Graphisch dargestellt.

\begin{figure}[H]
  \centering
  \includegraphics{plot1.pdf}
  \caption{Graphische Darstellung der Kennlinien.}
  \label{abb:6}
\end{figure}

Aus der graphischen Darstellung der Messwerte lässt sich nun der Sättigungsstrom
$I_s$ ablesen. Der Sättigungsstrom für die einzelnen Stromstärken sind in Tabelle
\ref{tab:2} dargestellt.

\begin{table}[H]
  \centering
  \caption{Sättigungsstrom für die einzelnen Stromstärken.}
  \label{tab:2}
  \begin{tabular}{c c}
    \toprule
    $I \, / \, A$ & $I_s \, / \, mA$ \\
    \midrule
    1,8 & 0,018  \\
    1,9 & 0,044  \\
    2,0 & 0,102  \\
    2,2 & 0,438  \\
    2,4 & 1,550  \\
    \bottomrule
  \end{tabular}
\end{table}

Als nächstes wird das Raumladungsgebiet für die maximale Heizleistung, für $\SI{2.4}{\ampere}$,
untersucht. Dazu werden die Messwerte aus Tabelle \ref{tab:1} aus dem Raumladungsgebiet
Graphisch dargestellt, in der Abbildung \ref{abb:7}. Das Raumladungsgebiet geht bis $\SI{60}{\volt}$.
Mit diesen Messwerten wird nun eine Ausgleichsrechnung durchgeführt mit Python 3.6.
Die verwendete Gleichung dabei lautet

\begin{equation*}
  I(U) = a \cdot U^b.
\end{equation*}

Es wird der Exponent der Strom-Spannung Beziehung bestimmt und mit der Gleichung
\ref{eq:2} verglichen.

\begin{figure}[H]
  \centering
  \includegraphics{plot2.pdf}
  \caption{Darstellung des Raumladungsgebiets und der Ausgleichsrechnung.}
  \label{abb:7}
\end{figure}

Damit ergibt sich der Exponent zu

\begin{equation*}
  b = \num{1.165(3)}.
\end{equation*}

Das entspricht einer Abweichung zu dem Langmuir-Schottkyschen Raumladungsgesetz von $22,33 \%$.

Daraufhin wird das Anlaufstromgebiet untersucht. Die Messwerte dieser Messung sind
in der Tabelle \ref{tab:3} dargestellt.

\begin{table}[H]
  \centering
  \caption{Messwerte für das Anlaufstromgebiet.}
  \label{tab:3}
  \begin{tabular}{c c}
    \toprule
    $ U \, / \, V$ & $ I \, / \, nA$ \\
    \midrule
    0,0 & 135,0 \\
    0,1 & 105,0 \\
    0,2 &  79,0 \\
    0,3 &  56,0 \\
    0,4 &  40,0 \\
    0,5 &  20,0 \\
    0,6 &  18,5 \\
    0,7 &  11,5 \\
    0,8 &   7,0 \\
    0,9 &   4,4 \\
    1,0 &   2,6 \\
    \bottomrule
  \end{tabular}
\end{table}

Mit Hilfe dieser Messwerte wird nun die Temperatur $T$ der Kathode bestimmt.
Dazu wird Gleichung \ref{eq:1} verwendet um eine Ausgleichsrechnung durchzuführen.
Diese Ausgleichsrechnung wird wieder mit Python 3.6 durchgeführt. Die Ergebnisse sind
in Abbildung \ref{abb:8} dargestellt.

\begin{figure}[H]
  \centering
  \includegraphics{plot3.pdf}
  \caption{Darstellung der Messwerte des Anlaufstromgebiet und der Ausgleichsrechnung.}
  \label{abb:8}
\end{figure}

Die Temperatur ergibt sich somit zu

\begin{equation*}
  T = \SI{}{\kelvin}.
\end{equation*}

Nun wird die Temperatur bei denen die einzelnen Kennlinien bestimmt wurden mittels
einer Leistungsbilanz bestimmt. Die aus dem Energiesatz hergeleitete Gleichung lautet

\begin{equation*}
  I_f U_f = f \eta \sigma T^4 + N_{WL}.
\end{equation*}

Dabei ist $f$ die emittierende Kathodenoberfläche $f = \SI{0.32}{\centi\meter\squared}$,
$\eta$ der Emissionsgrad $\eta = 0,28$, $\sigma$ ist die Strahlungskonstante $\sigma = \SI{5.7e-12}{\watt\per\centi\meter\kelvin\tothe{4}}$
und $N_{WL}$ ist die Wärmeleitung, die hier zu $N_{WL} = \SI{1}{\watt}$ abgeschätzt wird.

Die Messwerte von Strom und Spannung bei den einzelnen Kennlinien, sowie die
bestimmte Temperatur sind in Tabelle \ref{tab:4} gezeigt.

\begin{table}[H]
  \centering
  \caption{Darstellung der Messwerte von Strom und Spannung, sowie der Temperatur bei den einzelnen Kennlinien.}
  \label{tab:4}
  \begin{tabular}{c c c}
    \toprule
    $U \, / \, V$ & $I \, / \, A$ & $T \, / \, K$ \\
    \midrule
    1,8 & 3,9 & 1852,91 \\
    1,9 & 4,0 & 1896,01 \\
    2,0 & 4,4 & 1976,87 \\
    2,2 & 5,0 & 2103,56 \\
    2,4 & 5,9 & 2253,04 \\
    \bottomrule
  \end{tabular}
\end{table}


Als letzte wird noch die Austrittsarbeit des Kathodenmaterials bestimmt. Dafür wird
die Gleichung \ref{eq:3} nach der Austrittsarbeit $\Phi$ umgeformt

\begin{equation*}
  \Phi = - \frac{k_b T}{e_0} \cdot \ln \left(\frac{I_s h^3}{4 \pi e_0 m_0 k_b^2 T^2}\right).
\end{equation*}

Die Ergebnisse für die einzelnen Kennlinien sind in Tabelle \ref{tab:5} dargestellt.

\begin{table}[H]
  \centering
  \caption{Darstellung des Sättigungsströme, der Temperaturen und die Austrittsarbeiten.}
  \label{tab:5}
  \begin{tabular}{c c c}
    \toprule
    $I_s \, / \, mA$ & $T \, / \, K$ & $\Phi \, / \, eV$ \\
    \midrule
    0,018 & 1892,91 & 4,546 \\
    0,044 & 1896,01 & 4,513 \\
    0,102 & 1976,87 & 4,577 \\
    0,438 & 2103,56 & 4,628 \\
    1,550 & 2253,04 & 4,738 \\
    \bottomrule
  \end{tabular}
\end{table}

Aus den bestimmten Austrittsarbeiten wird nun der Mittelwert und die Standartabweichung
mit den folgenden Gleichungen bestimmt:

\begin{equation*}
    \bar{\Phi} = \frac{1}{5} \sum_{i=1}^{5} \Phi_i
\end{equation*}
\begin{equation*}
  \Delta \bar{\Phi} = \frac{1}{\sqrt{5}\sqrt{4}} \sqrt{\sum_{i=1}^{4}(\Phi_i-\bar{\Phi})^2}
\end{equation*}


Damit ergibt sich die gemittelte Austrittsarbeit zu

\begin{equation*}
  \bar{\Phi} = \SI{4.600(40)}{\eV}.
\end{equation*}
