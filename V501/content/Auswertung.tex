\section{Auswertung}
\subsection{Ablenkung im elektrischen Feld}
Es werden die Messdaten Strahlablenkung $D$, Plattenspannung $U_d$ und die Beschleunigungsspannung $U_B$ notiert.
Sie sind in der Tabelle \ref{tab:1} dargestellt.
\begin{table}[H]
  \centering
  \caption{Darstellung der Messwerte.}
  \label{tab:1}
  \begin{tabular}{c c c c c c}
\toprule
& \multicolumn{1}{c}{$U_B=\SI{200}{\volt}$} & \multicolumn{1}{c}{$U_B=\SI{250}{\volt}$} &\multicolumn{1}{c}{$U_B=\SI{300}{\volt}$}&\multicolumn{1}{c}{$U_B=\SI{400}{\volt}$}&\multicolumn{1}{c}{$U_B=\SI{500}{\volt}$}\\
\cmidrule(lr){2-2}\cmidrule(lr){3-3}\cmidrule(lr){4-4}\cmidrule(lr){5-5}\cmidrule(lr){6-6}
$D \, / \, cm$ & $U_d \, / \, V$ & $U_d \, / \, V$ & $U_d \, / \, V$ &$U_d \, / \, V$ & $U_d \, / \, V$\\
\midrule
5.080 & 21,9  & 28,9  & 33,8  & -    & -   \\
4.445 & 19,1  & 24,6  & 29,0  & -    & -   \\
3.810 & 14,9  & 19,6  & 23,3  & 30,7 & -   \\
3.175 & 11,2  & 14,6  & 17,2  & 22,9 & 31,3\\
2.540 &  6,5  &  8,5  & 10,2  & 14,7 & 21,2\\
1.905 &  2,5  &  4,5  &  4,7  &  6,1 & 10,2\\
1.270 & -1,6  & -0,7  & -1,6  & -1,0 & -0,5\\
0.635 & -5,3  & -5,9  & -7,4  & -9,2 &-11,3\\
0.000 & -9,3  &-10,6  &-13,9  &-17,0 &-21,6\\
\bottomrule
  \end{tabular}
\end{table}
Nun werden die Parameter $D$ und $U_d$ gegeneinander aufgetragen.
Dies ist in der Abbildung \ref{abb:6} dargestellt.
\begin{figure}[H]
  \centering
  \includegraphics{plot1.pdf}
  \caption{Darstellung der Messdaten.}
  \label{abb:6}
\end{figure}
Die lineare Ausgleichsrechnung wurde mit Python 3.6 durchgeführt.
Dazu wurde die Gleichung \ref{eq:1} verwendet:
\begin{equation*}
  \underbrace{D}_{\mathclap{y}} = \underbrace{\frac{pL}{2dU_B}}_{\mathclap{m}} \cdot \underbrace{U_d}_{\mathclap{x}} + \, b
\end{equation*}
Die folgenden Steigungen $m_{1-5}$ sind in der Tabelle \ref{tab:2} dargestellt.

\begin{table}[H]
  \centering
  \caption{Ergebnis der Ausgleichsrechnung.}
  \label{tab:2}
  \begin{tabular}{c c c}
\toprule
&$\text{Steigung} \,/\, \si{\centi\meter\per\volt}$& $U_B \, /\, \si{\volt}$\\
\midrule
$m_1$ & $\num{0.1587(22)}$& 200\\
$m_2$ & $\num{0.1268(13)}$& 250\\
$m_3$ & $\num{0.1051(12)}$& 300\\
$m_4$ & $\num{0.0797(6)}$ & 400\\
$m_5$ & $\num{0.0596(3)}$ & 500\\
\bottomrule
  \end{tabular}
\end{table}
Die bestimmten Steigungen sind die Empfindlichkeiten $\frac{D}{U_d}$ der Kathodenstrahlröhre
Diese werden nun gegen $\frac{1}{U_B}$ aufgetragen.
Die Darstellung ist in Abbildung \ref{abb:7} zu sehen.
\begin{figure}[H]
  \centering
  \includegraphics[width=\textwidth]{plot2.pdf}
  \caption{Darstellung zur Bestimmung der Konstruktionskonstanten $a$.}
  \label{abb:7}
\end{figure}
Auch hier wurde eine lineare Ausgleichsrechnung mit Python 3.6 durchgeführt.
Dazu wurde die Gleichung \ref{eq:1} erneut verwendet:
\begin{equation*}
  \underbrace{\frac{D}{U_d}}_{\mathclap{y}} = \underbrace{\frac{pL}{2d}}_{\mathclap{a}} \cdot \underbrace{\frac{1}{U_B}}_{\mathclap{x}} + b
\end{equation*}
Es folgt somit für die Konstruktionskonstante $a$
\begin{equation*}
  a = \SI{32,59(72)}{\centi\meter}.
\end{equation*}
Der theoretische Wert kann mithilfe der Abbildung \ref{abb:5} bestimmt werden.
Dabei ist
\begin{itemize}
  \item $p = \SI{1.9}{\centi\meter}$
  \item $L = \SI{15.33}{\centi\meter}$
  \item $d = \SI{0.665}{\centi\meter}$
\end{itemize}
Es folgt somit für die Konstruktionskonstante
\begin{equation*}
  a_{\text{Theo}} = \frac{pL}{2d} = \SI{21.9}{\centi\meter}
\end{equation*}

Es wird nun die Frequenz der Sinusspannung mithilfe der Sägezahnspannung
berechnet.
Dazu wird die Gleichung \ref{eq:2} genutzt. Dabei ist $m = 1$ und der
Proportionalitätsfaktor $n= 0.5, 1, 2, 3$
Die Messwerte werden in die Gleichung \ref{eq:2} eingesetzt und somit wird die Sinusfrequenz berechnet.
Die Ergebnisse sind in der Tabelle \ref{tab:3} dargestellt.
\begin{table}[H]
  \centering
  \caption{Ergebnisse zur Bestimmung der Sinusspannung.}
  \label{tab:3}
  \begin{tabular}{c c c}
\toprule
$\nu_{Sä} \,/\, \si{\hertz}$& $\text{Proportionalitätsfaktor} n$ & $\nu_{Si} \, /\, \si{\hertz}$\\
\midrule
29,33 & 3 & 87,99\\
49,99 & 2 & 99,98\\
79,32 & 1 & 79,32\\
158,64& 0,5& 79,32\\
\bottomrule
  \end{tabular}
\end{table}
Die Ergebnisse werden nun gemittelt.
Für den Mittelwert sowie die Standardabweichung werden die folgenden Gleichungen verwendet:
\begin{equation*}
    \bar{\nu}_{Si}= \frac{1}{N} \sum_{i=1}^{N} \nu_{Si,i}
\end{equation*}

\begin{equation*}
  \Delta \bar{\nu}_{Si} = \frac{1}{\sqrt{N}\sqrt{N-1}} \sqrt{\sum_{i}(\nu_{Si,i}-\bar{\nu}_{si})^2}
\end{equation*}
verwendet.
Es folgt für die Frequenz der Sinusspannung:
\begin{equation*}
  \nu_{si}= \SI{86.65(488)}{\hertz}
\end{equation*}
Des Weiteren soll der Scheitelwert mithilfe der Strahlablenkung $D$ und der Konstruktionskonstante $a$ berechnet werden.
Es wurde eine Beschleunigungsspannung $U_b = \SI{300}{\volt}$ angelegt. Die gemessene Konstruktionskonstante ist
$a = \SI{32,59(72)}{\centi\meter}$.
Bei den Messungen ergab sich eine konstante Strahlablenkung von $D = \SI{3.81}{\centi\meter}$.
Nun kann die Gleichung \ref{eq:1} umgeformt werden nach  der Plattenspannung $U_d$.
Es folgt:
\begin{equation*}
  U_d = \frac{D U_B}{a} = \SI{35.1(8)}{\volt}
\end{equation*}
Der Fehler lässt sich mithilfe der Gauß´schen Fehlerfortpflanzung wie gefolgt bestimmen
\begin{equation}
  \Delta U_d = \sqrt{\Bigl(-\frac{DU_B}{a^2}\cdot \Delta a\Bigr)^2}
\end{equation}
\subsection{Ablenkung im magnetischen Feld}
Um die spezifische Ladung der Elektronen bestimmen zu können, werden zunächst
die Messwerte der Ablenkung $D$, den Strom $I$, der durch die Helmholzspule geht, sowie die
Beschleunigungsspannung $U_B$ notiert.
Die Ergebnisse sind in der Tabelle \ref{tab:4} dargestellt.
\begin{table}[H]
  \centering
  \caption{Darstellung der Messergebnisse.}
  \label{tab:4}
  \begin{tabular}{c c c c c c}
\toprule
& \multicolumn{1}{c}{$U_B=\SI{200}{\volt}$} & \multicolumn{1}{c}{$U_B=\SI{250}{\volt}$} &\multicolumn{1}{c}{$U_B=\SI{300}{\volt}$}&\multicolumn{1}{c}{$U_B=\SI{400}{\volt}$}&\multicolumn{1}{c}{$U_B=\SI{500}{\volt}$}\\
\cmidrule(lr){2-2}\cmidrule(lr){3-3}\cmidrule(lr){4-4}\cmidrule(lr){5-5}\cmidrule(lr){6-6}
$D \, / \, cm$ & $I \, / \, A$ & $I \, / \, A$ & $I \, / \, A$ &$I \, / \, A$ & $I \, / \, A$\\
\midrule
0,635 & 0,30  & 0,30 & 0,40 & 0,40& 0,40\\
1,270 & 0,60  & 0,65 & 0,75 & 0,75& 0,80\\
1,905 & 0.95  & 1,00 & 1,15 & 1,15& 1,20\\
2,540 & 1,25  & 1,40 & 1,50 & 1,60& 1,65\\
3,175 & 1,60  & 1,62 & 1,90 & 2,00& 2,05\\
3,810 & 1,95  & 2,10 & 2,30 & 2,40& 2,50\\
4,445 & 2,25  & 2,50 & 2,70 & 2,85& 3,00\\
5,080 & 2,65  & 2,85 & 3,15 & 3,30& 3,40\\
\bottomrule
  \end{tabular}
\end{table}
Für das magnetische Feld $B$ wird die Gleichung \ref{eq:4} verwendet. Dabei ist die
Windungszahl $N = 20 $ und der Spulenradius $R = \SI{28.2}{\centi\meter}$.
Außerdem wird $\frac{D}{L^2+D^2}$ für $L = \SI{17.5}{\centi\meter}$ bestimmt, um später
mit der Gleichung \ref{eq:3} eine Ausgleichsrechnung durchführen zu können.
Die Ergebnisse sind in der Tabelle \ref{tab:5} dargestellt.
\begin{table}[H]
  \centering
  \caption{Darstellung der wichtigen Ergebnisse.}
  \label{tab:5}
  \begin{tabular}{c c c c c c}
\toprule
& \multicolumn{1}{c}{$U_B=\SI{200}{\volt}$} & \multicolumn{1}{c}{$U_B=\SI{250}{\volt}$} &\multicolumn{1}{c}{$U_B=\SI{300}{\volt}$}&\multicolumn{1}{c}{$U_B=\SI{400}{\volt}$}&\multicolumn{1}{c}{$U_B=\SI{500}{\volt}$}\\
\cmidrule(lr){2-2}\cmidrule(lr){3-3}\cmidrule(lr){4-4}\cmidrule(lr){5-5}\cmidrule(lr){6-6}
$\frac{D}{L^2+D^2} \, /\, \frac{1}{m}$ & $B \,/\, mT$ & $B \,/\, mT$ &$B \,/\, mT$ &$B \,/\, mT$ &$B \,/\, mT$\\
\midrule
0,31 & 0,01913 & 0,01913 & 0,02551 & 0,02551 & 0,02551\\
0,62 & 0,03826 & 0,04145 & 0,04783 & 0,04783 & 0,05102\\
0,92 & 0,06058 & 0,06377 & 0,07334 & 0,07334 & 0,07653\\
1,20 & 0,07971 & 0,08928 & 0,09566 & 0,10203 & 0,10522\\
1,48 & 0,10203 & 0,10331 & 0,12117 & 0,12754 & 0,13073\\
1,74 & 0,12117 & 0,13392 & 0,14667 & 0,15305 & 0,15943\\
1,98 & 0,14349 & 0,15943 & 0,17218 & 0,18175 & 0,19131\\
2,21 & 0,16899 & 0,18175 & 0,20088 & 0,21045 & 0,21682\\
\bottomrule
  \end{tabular}
\end{table}
Nun werden die Daten aus der Tabelle \ref{tab:5} gegeneinander aufgetragen und
in der Abbildung \ref{abb:8} graphisch dargestellt.
\begin{figure}[H]
  \centering
  \includegraphics[width=\textwidth]{plot3.pdf}
  \caption{Darstellung der Messergebnisse.}
  \label{abb:8}
\end{figure}
Mit Python 3.6 wurde eine lineare Ausgleichsrechnung durchgeführt.
Dazu wurde die Gleichung \ref{eq:3} verwendet:
\begin{equation*}
  \underbrace{\frac{D}{L^2+D^2}}_{\mathclap{y}} = \underbrace{\frac{1}{\sqrt{8U_B}} \sqrt{\frac{e_0}{m_0}}}_{\mathclap{m}} \cdot \underbrace{B}_{\mathclap{x}} + b
\end{equation*}
Die Ergebnisse für die Bestimmung der spezifische Ladung $\frac{e_0}{m_0}$ werden
in der Tabelle \ref{tab:6} dargestellt. Dabei wurde folgende Rechnung verwendet:
\begin{equation*}
  \frac{e_0}{m_0} = 8 \cdot U_B \cdot m_{1-5}^2
\end{equation*}
Der Fehler lässt sich mit der Gauß´schen Fehlerfortpflanzung bestimmen
\begin{equation*}
  \Delta \frac{e_0}{m_0} = \sqrt{(16 \cdot U_b \cdot m \cdot \Delta m)^2}.
\end{equation*}


\begin{table}[H]
  \centering
  \caption{Darstellung der spezifische Ladung.}
  \label{tab:6}
  \begin{tabular}{c c c c}
\toprule
$\text{Beschleunigungsspannung}\, U_B \,/\, V$ & $\text{Steigung m} \,/\, \frac{1}{m\cdot mT}$ &$\frac{e_0}{m_0} \,/\, 10^{11}\frac{C}{kg}$ & $\text{Abweichung} \,/\, \%$\\
\midrule
250 &$\num{8.93(23)}$ &$\num{1.59(8)}$ &  9,66\\
300 &$\num{8.12(25)}$ &$\num{1.58(10)}$& 10,23\\
350 &$\num{7.58(19)}$ &$\num{1.61(8)}$ &  8,52\\
400 &$\num{6.97(64)}$ &$\num{1.55(29)}$& 11,93\\
430 &$\num{6.69(64)}$ &$\num{1.54(30)}$& 12,50\\
\bottomrule
  \end{tabular}
\end{table}

Aus den bestimmten spezifischen Ladungen wird nun der Mittelwert und die Standardabweichung
bestimmt mit den folgenden Gleichungen.

\begin{equation*}
  \overline{\frac{e_0}{m_0}}= \frac{1}{N} \sum_{i=1}^{N} \frac{e_0}{m_0}_{i}
\end{equation*}
\begin{equation*}
\Delta \overline{\frac{e_0}{m_0}} = \frac{1}{\sqrt{N}\sqrt{N-1}} \sqrt{\sum_{i}\left(\frac{e_0}{m_0}_{i}-\overline{\frac{e_0}{m_0}}\right)^2}
\end{equation*}

Der Mittelwert ergibt sich somit zu:

\begin{equation*}
  \overline{\frac{e_0}{m_0}} = \SI{1.574(12)e11}{\coulomb\per\kilo\gram}
\end{equation*}

Zuletzt soll die Totalintensität des Erdmagnetfeldes errechnet werden.
Aus dem Spulenstrom kann das Magnetfeld, was in der Helmholzspule erzeugt wird,
berechnet werden.
Mit einem gemessenen Spulenstrom von
\begin{equation*}
  I_H =\SI{126}{\milli\ampere}
\end{equation*}
folgt mit Gleichung \ref{eq:4} für die Magnetfeldstärke:
\begin{equation*}
  B_H =\SI{8.035}{\micro\tesla}.
\end{equation*}
Zu beachten ist, dass die Magnetfeldstärke $B_H$ die horizontale Komponente des Erdmagnetfelds ist.
Wichtig hierbei ist unter welchem Winkel sich das Erdmagnetfeld ausrichtet.
Die gemessenen Winkeln zwischen Erdmagnetfeld und der horizontalen Komponente des Magnetfeldes
sind in der Tabelle \ref{tab:7} dargestellt.
\begin{table}[H]
  \centering
  \caption{Messung des Winkels an verschiedenen Stellen.}
  \label{tab:7}
  \begin{tabular}{c}
\toprule
$\text{Winkel}\, \Phi \,/\, \circ$\\
\midrule
82\\
73\\
55\\
55\\
\bottomrule
  \end{tabular}
\end{table}
Mit den folgenden Gleichungen wird der Mittelwert sowie die Standardabweichung berechnet.
\begin{equation*}
  \bar{\Phi}= \frac{1}{N} \sum_{i=1}^{N} \Phi_{i}
\end{equation*}
\begin{equation*}
\Delta \bar{\Phi} = \frac{1}{\sqrt{N}\sqrt{N-1}} \sqrt{\sum_{i}(\Phi_{i}-\bar{\Phi})^2}
\end{equation*}
Es folgt somit für den Winkel:
\begin{equation*}
  \Phi = \SI{66.25(675)}{\degree}
\end{equation*}
Aus der Überlegung, dass sich das Erdmagnetfeld in zwei Komponente zerlegen lässt, folgt für die Berechnung des Erdmagnetfelds
\begin{equation*}
  B = \frac{B_H}{cos(\Phi)}
\end{equation*}
Auch hier lässt sich der Fehler mithilfe der Gauß´schen Fehlerfortpflanzung berechnen
\begin{equation*}
  \Delta B = \sqrt{\Bigl(\frac{B_H sin(\Phi)}{cos^2(\Phi)} \cdot \Delta \Phi\Bigr)^2}
\end{equation*}
Somit ergibt sich für das Erdmagnetfeld:
\begin{equation*}
  B= \SI{20(5)}{\micro\tesla}
\end{equation*}
