\section{Diskussion}
Zur Bestimmung der Konstruktionskonstante $a$ wurde
folgender Wert gemessen
\begin{equation*}
    a = \SI{32,59(72)}{\centi\meter}
\end{equation*}
Der theoretische Wert der Konstruktionskonstante liegt bei
\begin{equation*}
   a_{\text{Theo}} = \SI{21.9}{\centi\meter}
\end{equation*}
Dies ist eine Abweichung von $48,81 \%$. Grund für eine solch hohe Abweichung
können durch Ablesefehler erfolgen. Des Weiteren war der Punkt, zur Ablesung der
Strahlablenkung, nicht genug fokussiert, um einen optimaleren Wert zu bekommen.
Ein weiter Grund ist, dass der errechnete Theoriewert falsch ist, da die Ausgleichsrechnung
in Abbildung \ref{abb:11} sehr gut ist. 
Bei der Sinusspannung kann die Aussage gemacht werden, dass sie in einem Frequenzbereich
von $\nu_{Theo}=80 - 90 \si{\hertz}$ gelegen hat.
Unser gemessener Wert lag bei
\begin{equation*}
  \nu_{si}= \SI{86.65(488)}{\hertz}
\end{equation*}
Dies deutet auf eine gute Messung hin.\\
Bei der Bestimmung der spezifische Ladung der Elektronen war der bestimmte Wert

\begin{equation*}
  \overline{\frac{e_0}{m_0}} = \SI{1.574(12)e11}{\coulomb\per\kilo\gram}.
\end{equation*}

Der theoretisch bestimmte Wert für die spezifische Ladung ist

\begin{equation*}
  \frac{e_0}{m_0}_\text{theo} = \SI{1.76e11}{\coulomb\per\kilo\gram}.
\end{equation*}

Das entspricht einer Abweichung von $10,57 \%$.

Wichtig hierbei ist, dass die spezifische Ladung der Elektronen in der selben Größenordnung liegt. Um eine bessere
Bestimmung der spezifische Ladung zu erhalten, wären mehrere Messungen nötig gewesen.\\
Bei der Bestimmung des Erdmagnetfeldes kam folgender Wert raus:
\begin{equation*}
  B= \SI{20(5)}{\micro\tesla}
\end{equation*}
Der Literaturwert kann mithilfe eines Rechners ziemlich genau bestimmt werden \cite{3}.
Die Totalintensität liegt bei $B_I =\SI{49,08}{\micro\tesla}$ in Dortmund.
Die horizontale Komponente liegt bei $B_{H, \text{lit}} = \SI{19.35}{\micro\tesla}$.
Der Arkuskosinus zwischen den beiden Werte ergibt einen Winkel von $\Phi_{\text{lit}} = \SI{66.78}{\degree}$.
Der gemessene Winkel für die Bestimmung des Erdmagnetfeldes liegt bei $\Phi_{\text{mess}} = \SI{66.25(675)}{\degree}$.
Die Abweichung von den beiden Winkeln liegt bei $0,79 \, \%$. Dies deutet auf eine gute Winkelmessung hin.
Hingegen bei der Berechnung von der Totalintensität eine Abweichung von ca $59,25 \%$ vorliegt.
Ebenso hat sich bei der gemessenen horizontalen Komponente, die $B_{H, \text{mess}} =\SI{8.035}{\micro\tesla}$ beträgt,
eine Abweichung von $58,48 \,\%$ von dem Literaturwert ergeben.
Es kann die Aussage gemacht werden, dass die Messung, die mithilfe der Helmholtz Spule
durchgeführt wurde, schlecht war. Das könnte daran liegen, dass andere Geräte,
die sich auch in dem Raum befanden, die Messung beeinflusst haben könnten. Andererseits
kann die Fehlerquelle bei dem Messvorgang liegen, da die Ausrichtung der Kathodenstrahlröhre
schwierig war, weil der verwendete Kompass ziemlich ungenau war.
