\section{Diskussion}
Zur Bestimmung der Konstruktionskonstante $a$ wurde
folgender Wert gemessen
\begin{equation*}
    a = \SI{32,59(72)}{\centi\meter}
\end{equation*}
Der theoretische Wert der Konstruktionskonstante liegt bei
\begin{equation*}
   a_{\text{Theo}} = \SI{35.75}{\centi\meter}
\end{equation*}
Dies ist eine Abweichung von $8,84 \, \%$. Grund für eine solch hohe Abweichung
können durch Ablesefehler erfolgen. Des Weiteren war der Punkt, zur Ablesung der
Strahlablenkung, nicht genug fokussiert gewesen, um einen optimaleren Wert zu bekommen.
Bei der Sinusspannung kann die Aussage gemacht werden, dass sie in einem Frequenzbereich
von $\nu_{Theo}=80 - 90 \si{\hertz}$ gelegen hat.
Unser gemessener Wert lag bei
\begin{equation*}
  \nu_{si}= \SI{86.65(488)}{\hertz}
\end{equation*}
Dies deutet auf eine gute Messung hin.\\
Bei der Bestimmung der spezifische Ladung der Elektronen werden die Ergebnisse in der Tabelle \ref{tab:8}
erneut dargestellt.
\begin{table}[H]
  \centering
  \caption{Darstellung der Ergebnisse.}
  \label{tab:8}
  \begin{tabular}{c c c c}
\toprule
$\frac{e_0}{m_0}_{\text{gemessen}}\,/\, 10^{11}\frac{C}{kg}$ & $\frac{e_0}{m_0}_{\text{Theorie}} \,/\, 10^{11}\frac{C}{kg}$& $\text{Abweichung} \,/\, \%$\\
\midrule
$\num{1.59(8)}$ &1,76 &  9,66\\
$\num{1.58(10)}$&1,76 & 10,23\\
$\num{1.61(8)}$ &1,76 &  8,52\\
$\num{1.55(29)}$&1,76 & 11,93\\
$\num{1.54(30)}$&1,76 & 12,50\\
\bottomrule
  \end{tabular}
\end{table}
Wichtig hierbei ist, dass die spezifische Ladung der Elektronen in der selben Größenordnung liegt. Um eine bessere
Bestimmung der spezifische Ladung zu erhalten, wären mehrere Messungen nötig gewesen.
Bei der Bestimmung des Erdmagnetfeldes kam folgender Wert raus:
\begin{equation*}
  B= \SI{20(5)}{\micro\tesla}
\end{equation*}
Der Literaturwert \cite{3} liegt bei ca. $B_{lit} \approx \SI{40}{\micro\tesla}$.
Dies ist eine Abweichung von $50 \, \%$. Diese hohe Abweichung kam dadurch zustande, dass die Winkelmessungen
zu ungenau waren. Ebenso war die Apparatur nicht ganz funktionstüchtig, sodass gute Messergebnisse erzielt werden
können.
