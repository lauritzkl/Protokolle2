\section{Diskussion}
Zur Bestimmung der Konstruktionskonstante $a$ wurde
folgender Wert gemessen
\begin{equation*}
    a = \SI{32,59(72)}{\centi\meter}
\end{equation*}
Der theoretische Wert der Konstruktionskonstante liegt bei
\begin{equation*}
   a_{\text{Theo}} = \SI{35.75}{\centi\meter}
\end{equation*}
Dies ist eine Abweichung von $8,84 \, \%$. Grund für eine solch hohe Abweichung
können durch Ablesefehler erfolgen. Des Weiteren war der Punkt, zur Ablesung der
Strahlablenkung, nicht genug fokussiert gewesen, um einen optimaleren Wert zu bekommen.
Bei der Sinusspannung kann die Aussage gemacht werden, dass sie in einem Frequenzbereich
von $\nu_{Theo}=80 - 90 \si{\hertz}$ gelegen hat.
Unser gemessener Wert lag bei
\begin{equation*}
  \nu_{si}= \SI{86.65(488)}{\hertz}
\end{equation*}
Dies deutet auf eine gute Messung hin.\\
Bei der Bestimmung der spezifische Ladung der Elektronen werden die Ergebnisse in der Tabelle \ref{tab:8}
erneut dargestellt.
\begin{table}[H]
  \centering
  \caption{Darstellung der Ergebnisse.}
  \label{tab:8}
  \begin{tabular}{c c c c}
\toprule
$\frac{e_0}{m_0}_{\text{gemessen}}\,/\, 10^{11}\frac{C}{kg}$ & $\frac{e_0}{m_0}_{\text{Theorie}} \,/\, 10^{11}\frac{C}{kg}$& $\text{Abweichung} \,/\, \%$\\
\midrule
$\num{1.59(8)}$ &1,76 &  9,66\\
$\num{1.58(10)}$&1,76 & 10,23\\
$\num{1.61(8)}$ &1,76 &  8,52\\
$\num{1.55(29)}$&1,76 & 11,93\\
$\num{1.54(30)}$&1,76 & 12,50\\
\bottomrule
  \end{tabular}
\end{table}
Wichtig hierbei ist, dass die spezifische Ladung der Elektronen in der selben Größenordnung liegt. Um eine bessere
Bestimmung der spezifische Ladung zu erhalten, wären mehrere Messungen nötig gewesen.
Bei der Bestimmung des Erdmagnetfeldes kam folgender Wert raus:
\begin{equation*}
  B= \SI{20(5)}{\micro\tesla}
\end{equation*}
Der Literaturwert \cite{3} kann mithilfe eines Calculators ziemlich genau bestimmt werden.
Die Totalintensität liegt bei $B_I =\SI{49,08}{\micro\tesla}$ in Dortmund.
Die horizontale Komponente liegt bei $B_{H_lit} = \SI{19.35}{\micro\tesla}$.
Der Arkuskosinus zwischen den beiden Werte erigbt einen Winkel von $\Phi_{lit} = \SI{66,78}{\degree}$.
Unser gemessener Winkel für die Bestimmung des Erdmagnetfeldes liegt bei $\Phi_{mess} = \SI{66.25(675)}{\degree}$.
Die Abweichung von den beiden Winkeln liegt bei $0,79 \, \%$. Dies deutet auf eine gute Winkelmessung hin.
Hingegen bei der Berechnung von der Totalintensität eine Abweichung von ca $59,25 \,\%$ vorliegt.
Ebenso ist bei der horizontalen Komponente, was bei einem gemessenen Wert von $\B_{H_mess} =\SI{8.035}{\micro\tesla}$ war,
eine Abweichung von $58,48 \,\%$ vom Literaturwert.
Es kann die Aussge gemacht werden, dass die Messung, mithilfe der Helmholzspule, schlecht war. Es wären mehrere Messungen
nötig gewesen um den Fehler zu minimieren.  
