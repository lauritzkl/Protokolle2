\section{Durchführung}
Der Versuchsaufbau zur Messung der Wellenlänge als auch
zur Bestimmung des Brechungsindex von Luft ist in Abbildung \ref{abb:3}
dargestellt.
\begin{figure}[H]
  \includegraphics[width=\textwidth]{content/Aufbau3.png}
  \caption{Aufbau zur Bestimmung von Wellenlänge und Brechungsindex.\cite{1}}
  \label{abb:3}
\end{figure}
Bevor die Messung beginnt, wird zunächst das Michelson-Interferometer justiert.
Dazu wird mithilfe einer Mattscheibe vor dem Detektor $D$ die Strahlen aufgefangen.
Mit dem verstellbaren Spiegel wird versucht das die Strahlen möglichst aufeinander liegen.
Um die Interferenz besser sehen zu können wird vor dem Laser eine Sammellinse platziert.
Der Detektor wird anschließend noch auf die richtige Höhe gebracht, sodass die Zählung
der Helligkeitsmaxima möglich ist.
Nun wird ein verschiebarer Spiegel mithilfe eines Moters um Mikrometer verschoben.
Dabei ist zu achten, dass der Spiegel sich nicht zu schnell bewegt, da sonst der
Detektor nicht alle Helligkeitsmaxima aufzeichnen kann.
Es wird nun die Länge $d_0$ aufgezeichnet sowie die Anzahl der Maxima. Um möglichst
genau die Wellenlänge des Lasers bestimmen zu können wird der Versuch achtmal durchgeführt.
Um den Brechungsindex von Luft zu messen, wird die Platte evakuiert und dessen Druck aufgeschrieben.
Dabei werden die Spiegeln nicht verstellt. Beim wieder einlassen der Luft entstehen Helligkeitsmaxima
und diese werden notiert.
