\section{Auswertung}

Zunächst wird die Wellenlänge des Lasers bestimmt. Dazu wird die Gleichung
\ref{eq:} verwendet. Dabei ist $\Delta d_0 = \SI{5}{\milli\meter}$. Allerdings muss
noch der Hebel-Untersetzungsfaktor $u = \frac{1}{5,017}$ bei der Vergrößerung $\Delta d_0$
beachtet werden. Damit ergibt sich für die Vergrößerung $\Delta d = \SI{0.997}{\milli\meter}$.
Die Messwerte und die bestimmten Wellenlängen sind in Tabelle \ref{tab:1} dargestellt.

\begin{table}[H]
  \centering
  \caption{Darstellung der Messwerte und der Wellenlängen.}
  \label{tab:1}
  \begin{tabular}{c c}
    \toprule
    $z$ & $\lambda \, / \, \si{\nano\meter}$ \\
    \midrule
    3376 & 590 \\
    3065 & 650 \\
    3100 & 643 \\
    3007 & 663 \\
    3001 & 664 \\
    3039 & 656 \\
    3112 & 640 \\
    3025 & 659 \\
    \bottomrule
  \end{tabular}
\end{table}

Nun wird der Mittelwert und die Standartabweichung mit den Gleichungen \ref{e:1} und
\ref{e:2} bestimmt.


\begin{equation}
    \bar{\lambda} = \frac{1}{8} \sum_{i=1}^{8} \lambda_i
    \label{e:1}
\end{equation}
\begin{equation}
  \Delta \bar{\lambda} = \frac{1}{\sqrt{8}\sqrt{7}} \sqrt{\sum_{i=1}^{8}(\lambda_i-\bar{\lambda})^2}
  \label{e:2}
\end{equation}

Somit ergibt sich für die gemittelte Wellenlänge

\begin{equation*}
  \bar{\lambda} = \SI{646(8)}{\nano\meter}.
\end{equation*}

Daraufhin wird der Brechungsindex von Luft bestimmt. Die Messwerte sind in Tabelle
\ref{tab:2} dargestellt. Aus diesen Messwerten wird zunächst mit Hilfe der Gleichung
\ref{eq:} $\Delta n$ bestimmt. Bei der Gleichung ist $\lambda = \SI{635}{\nano\meter}$.
Mit dem $\Delta n$ wird nun mit der Gleichung \ref{eq:} $n$ bestimmt.

\begin{table}[H]
  \centering
  \caption{Darstellung der Messwerte für den Brechungsindex und den errechneten Brechungsindex.}
  \label{tab:2}
  \begin{tabular}{c c}
    \toprule
    $z$ & $n$ \\
    \midrule
    34 & $\num{1.000273}$ \\
    33 & $\num{1.000265}$ \\
    31 & $\num{1.000249}$ \\
    32 & $\num{1.000257}$ \\
    31 & $\num{1.000249}$ \\
    32 & $\num{1.000257}$ \\
    33 & $\num{1.000265}$ \\
    32 & $\num{1.000257}$ \\
    \bottomrule
  \end{tabular}
\end{table}

Nun werden die errechneten Brechungsindices wieder gemittelt und es wird die
Standartabweichung bestimmt mit den Gleichungen \ref{e:3} und \ref{e:4}.

\begin{equation}
    \bar{n} = \frac{1}{8} \sum_{i=1}^{8} n_i
    \label{e:3}
\end{equation}
\begin{equation}
  \Delta \bar{n} = \frac{1}{\sqrt{8}\sqrt{7}} \sqrt{\sum_{i=1}^{8}(n_i-\bar{n})^2}
  \label{e:4}
\end{equation}

Damit ergibt sich für den Brechungsindex von Luft

\begin{equation*}
  \bar{n} = \num{1.000259(3)}.
\end{equation*}
