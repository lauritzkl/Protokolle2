\section{Diskussion}

Bei dem ersten Aufgabenteil, der Vorbereitung, ist die Abweichung zwischen der gemessenen
Länge und der mit dem Ultraschall bestimmten Länge des Zylinders $0,38 \%$. Diese Abweichung
liegt im Toleranzbereich.

Bei dem Impuls-Echo-Verfahren ist die bestimmte Schallgeschindigkeit $c_e = \SI{2709.65(486)}{\meter\per\second}$.
Mit dem Theoriewert von $c = \SI{2730}{\meter\per\second}$ ergibt sich eine Abweichung von
$0.75 \%$ was auch im Toleranzbereich liegt.

Die mit dem Durchschallungs-Verfahren bestimmte Schallgeschindigkeit ist $c_d = \SI{2676.82(1168)}{\meter\per\second}$.
Dabei ist die Abweichung $1,95 \%$.

Auch die Abweichung der beiden Verfahren, die bei $1,21 \%$ liegt, ist im Toleranzbereich.

Alles in allem liegen alle Messergebnisse im Toleranzbereich, was auf eine gute
Messung mit wenig systematischen Fehlern schließen lässt. 
