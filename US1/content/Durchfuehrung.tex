\section{Durchführung}
Der experimentelle Aufbau besteht aus einem Ultraschallechoskop, einer Ultraschallsonde und einem Rechner für die Messaufnahmen.
Die Sende- sowie Empfangsleistung der Sonde liegt im Bereich von $30 - 35\, dB$. Für die Datenaufnahme wird der A-Scan verwendet.
Um eine klare Darstellung der Impulse am Rechner zu ermöglichen, wird zwischen dem Körper und der Sonde destilliertes Wasser aufgetragen.
Zunächst wird mit Hilfe des Puls-Echo-Verfahrens die Laufzeit zwischen zwei Impulsen eines Acrylzylinders bestimmt und aus den gemessenen
Daten wird die Schallgeschwindigkeit in Acryl berechnet, siehe Gleichung (\ref{eq:6}).
Anschließend wird mit einer Schieblehre die Länge des Acrylzylinders bestimmt.
Um den Absorptionskoeffizienten zu bestimmen, werden an sechs Acrylzylindern, unterschiedlicher Höhe, die Amplituden des ausgesendeten und des reflektierten Pulses aufgenommen.
Zur Schallgeschwindigkeitsbestimmmung von Acryl werden wieder an sechs Acrylzylindern mit zwei unterschiedlichen Messmethoden, einmal Implus-Echo-Verfahren sowie Durchschallungs-Verfahren,
die Laufzeiten aufgenommen.
\newline
Zur spektralen Analyse werden zwei Acrylscheiben mit bidestilliertem Wasser gekoppelt und anschließend unter einen Acrylzylinder gestellt.
Der Zylinder dient als Vorlaufstrecke, um ein Mehrfachecho besser von dem Initialecho trennen zu können. Bei dieser Aufnahme wird versucht möglichst drei
Mehrfachreflexionen darstellen zu können, um anschließend am Rechner ein Spektrum sowie Cepstrum zu erhalten. Mit diesem Sprektrum kann nun die Dicke
der beiden Scheiben bestimmt werden.
Zur letzten Messaufnahme wird ein Augenmodell mithilfe des Impuls-Echo-Verfahrens untersucht. Es muss beachtet werden, dass die Schallgeschwindigkeit in der Linse
$c_\text{L} = \SI{2500}{\metre\per\second}$ anders ist als die Geschwindigkeit im Glaskörper $c_\text{GK} = \SI{1410}{\metre\per\second}$.
