\section{Zielsetzung}
In diesem Versuch sollen die Grundlagen der Ultraschallechographie
in ihren Anwedungsbereichen kennengelernt werden.
\section{Theorie}
Als Ultraschall bezeichnet man den Frequenzbereich von 20 kHz bis ca. 1 GHz.
Zum Vergleich hören Menschen in eine Frequenzbereich von 16 Hz bis ca. 20 kHz.
Um die longitudinale elektromagnetische Welle, in seinen Anwedungen wie in der Werkstoffprüfung
und in der Medizin, beschreiben zu können kann diese Formel
\begin{equation}
  p(x,t)= p_0 + v_0 Z cos(\omega t - k x)
  \label{eq:1}
\end{equation}
angewendet werden.
Dabei ist $Z = c* \rho$ die akustische Impedanz. $\rho$ ist die Dichte des durchstrahlten Materials.
Da Ultraschall aufgrund von Druckschankungen sich fortbewegt, ist sie von Druck- bzw. Dichteänderung materialabhängig.
Die Schallgeschindigkeit in einer Flüssigkeit ist von ihrer Kompressibilität $\kappa$ und deren Dichte $\rho$
abhängig.
\begin{equation}
  c_{\text{flüssig}} = \sqrt{\frac{1}{\kappa \cdot \rho}}
  \label{eq:2}
\end{equation}
Bei Schallausbreitung in Festkörpern sind Transversalwellen möglich, da sie Richtungsabhängig sind.
Hierbei ist die Kompressibilität $\kappa$ antiproprtional zum Elastizitätsmodul $E$ des Festkörpers.
\begin{equation}
  c_{\text{fest}} = \sqrt{\frac{E}{\rho}}
  \label{eq:3}
\end{equation}
Die Energie durch die Absorption nimmt exponetiell ab.
\begin{equation}
  I(x) =I_0 \cdot e^{-\alpha x}
  \label{eq:4}
\end{equation}
$\alpha$ beschreibt den Absorptionskoeffizient der Schallamplitude.
Da Luft ein sehr hohe Absorptionskoeffizient hat wird in der Medizin ein Kontaktmittel
zwischen Sonde und dem untersuchendem Material verwendet.
Wie bei elektromagnetische Wellen haben sie die Eigenschaft zu reflektieren.
Dabei setzt sich der Reflexionskoeffizient R, das Verhältnis von reflektierten zu einfallender Schallintensität zusammen.
\begin{equation}
  R = (\frac{Z_1 - Z_2} {Z_1 + Z_2})^2
  \label{eq:5}
\end{equation}
Der Anteil der transmittierten Wellen lässt sich aus $T=1-R$ berechnen.
Um Ultraschallwellen zu erzeugen nutzt man den piezo-elektrischen Effekt aus.
Dabei werden die piezoelektrische Kristall in ein elektrisches Wechselfeld gebracht.
Durch die Richtung des elektrischen Feldes werden die polaren Achsen des Kristalls in Schwingung angeregt und
senden dabei Ultraschallwellen aus.
In der Medizin werden zwei Messmethoden angewendet.
Zum einen das Durchschallungs-Verfahren. Wie der Name schon sagt werden die Probestücke durchschallt. Dabei sendet
der Sender ein kurzeitigen Schallimpuls aus und am andere Ende wird dies durch ein Empfänger aufgefangen.
Bei einer Fehlstelle wird die Intensität abgeschwächte am Empfänger gemessen.
Die andere Messmethode ist das Impuls-Echo-Verfahren. Dabei ist der Sender gleichzeitig der Empfänger. Bei bekannter Schallgeschwindigkeit
bewegt sie sich mit dem Weg-Geschwindigkeits-Gesetzt
\begin{equation}
  s=\frac{1}{2} c t
  \label{eq:6}
\end{equation}
fort. Bei einer Fehlstelle wird der reflektierte Schall eher am Empfänger angekommen.
Die Laufzeitdiagramme können in einen A- oder B-Scan dargestellt werden.
