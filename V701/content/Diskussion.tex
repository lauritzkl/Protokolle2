\section{Diskussion}

Bei der ersten Messung zur Bestimmung der Reichweite der Strahlung fällt auf, dass
sich der Channel kaum ändert. Das kann daran liegen, dass bei der Messung der falsche Peak
gemessen wurde, der durch Untergrundstrahlung entstanden ist. Die Störstrahlung hat bei jeder Messung
die Messwerte beeinflusst.
Aus diesem Grund lässt sich der berechnete Energieverlust nicht mit den anderen Wert
verglichen.
Auch bei der ersten Messung der mittleren Reichweite wird bei der Abbildung \ref{abb:2}
sichtbar, dass die Zählraten nicht linear abnehmen und die lineare Ausgleichsrechnung
deshalb nicht sehr genau sein kann. Die bestimmte mittlere Reichweite beträgt
$R_m = \SI{1.30(26)}{\centi\meter}$ war einer Energie von $E_{Rm} = \SI{0.121(16)}{\mega\eV}$
entspricht. Diese Energie erfüllt die Voraussetzung für die Gleichung \ref{eq:3}.

Bei der zweiten Messung entsprechen die Messwerte eher den theoretischen Vorüberlegungen,
was an den linearen Ausgleichsrechnungen gut zu sehen ist. Die mittlere Reichweite
in diesem Fall beträgt $R_m = \SI{1.95(13)}{\centi\meter}$ was einer Abweichung
von $33,33 \%$ zu dem aus der ersten Messung bestimmten Wert entspricht.
Diese mittlere Reichweite entspricht einer Energie von $E_{Rm} = \SI{0.158(7)}{\mega\eV}$
was auch die Voraussetzung der Gleichung \ref{eq:3} entspricht.
Der Energieverlust bei dieser Messung liegt bei $-\frac{dE}{dx} = \SI{0.730(24)}{\mega\eV\per\centi\meter}$.

Für die Bestimmung der Statistik des radioaktiven Zerfalls wurden viel zu wenig
Messwerte aufgenommen. An dem Histogramm in Abbildung \ref{abb:6} wird deutlich,
dass die Messwerte weder einer Poissionverteilung, noch einer Gaußverteilung
entsprechen. Für die Statistik eines radioaktiven Zerfalls ist normalerweise eine Poissionverteilung
zu erwarten, da die Zerfälle zufällig und unabhängig voneinander stattfinden.
