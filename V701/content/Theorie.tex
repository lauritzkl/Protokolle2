\section{Zielsetzung}
In diesem Versuch wird die Reichweite von $\alpha$-Strahlung in Luft berechnet.
\section{Theorie}
Die Energie von $\alpha$-Strahlung kann durch Messen ihrer Reichweite bestimmt werden.
Die Rutherford Streuung beschreibt die Energieabgabe durch elastische Stöße von
$\alpha$-Teilchen mit Materie.
Neben Ionisationsprozessen können die $\alpha$-Teilchen durch Anregung oder
Dissoziation von Molekülen ihre Energie verlieren.
Der Energieverlust ist von der Energie der $\alpha$-Strahlung sowie der Dichte
des durchlaufenen Materials abhängig.
Für große Energien der $\alpha$-Teilchen kann die Bethe-Bloch-Gleichung
den Energieverlust beschreiben
\begin{equation*}
  -\frac{dE_{\alpha}}{dx} = \frac{z^2 e^4}{4 \pi \epsilon_0 m_e}\frac{nZ}{v^2} ln(\frac{2m_e v^2}{I})
  \label{eq:1}
\end{equation*}
Der Energieverlust $\frac{dE_{\alpha}}{dx}$ steigt mit Wahrscheinlichkeit zur Wechselwirkung mit
dem Material bei sehr kleinen Geschwindigkeiten.
Dabei beschreibt $z$ die Ladung und $v$ die Geschwindigkeit der $\alpha$-Strahlung.
$Z$ ist die Ordnungszahl, n die Teilchendichte und $I$ die Ionisierungsenergie
des Targetgases. Bei kleinen Energien verliert die Bethe-Bloch Gleichung ihre Gültigkeit,
da es zu Ladungsaustauschprozessen kommt.
Die Wegstrecke zur vollständigen Abbremsung
und somit auch die Reichweite $R$ des $\alpha$-Teilchens
kann so beschrieben werden
\begin{equation*}
  R = \int_{0}^{E_{\alpha}} \frac{dE_\alpha}{-\frac{dE_{\alpha}}{dx}}
  \label{eq:2}
\end{equation*}
Da die Energie $\propto $ zur Geschwindigkeit ist treten hier vermehrte Ladungsaustauschprozessen auf.
Für kleinere Enegien $E_\alpha \leq 2,5 MeV$ kann die mittlere Reichweite
\begin{equation}
  R_m= 3,1 \cdot \sqrt{E^2_\alpha}
  \label{eq:3}
\end{equation}
verwendet werden.
Die Reichweite in Gasen ist bei konstanter Temperatur und konstantem Volumen $\propto$ zum Druck $p$.

Mithilfe einer Absorptionsmessung kann der Druck $p$ variiert werden. Dazu
wird bei einem festen Abstand $x_0$ zwischen Detektor und $\alpha$-Strahler die
effektive Länge $x$ folgende Relation angenommen
\begin{equation}
  x = x_0 \frac{p}{p_0}
  \label{eq:4}
\end{equation}
Dabei ist $p_0 = 1013 \, mbar$ der Umgebungsdruck.
