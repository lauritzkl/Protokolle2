\section{Durchführung}
In der Abbildung (\ref{abb:2}) ist eine schematische Darstellung zur Messung von radioaktiven
Strahlen.
\begin{figure}[H]
  \centering
  \includegraphics[width=\textwidth}{Verlauf.png}
  \caption{Darstellung der Messaperatur \cite{1}.}
\end{figure}
Mit Hilfe einem Geiger-Müller-Zählrohr wird ein konstanter Bruchteil emittierte Strahlung nachgewiesen.




AlsersteswirdeineMessungfürdenNulleffekt𝑁u gestartet.
Dieserwirddurchkosmische Strahlung und natürliche Radioaktivität
hervorgerufen und muss für die Messung der Halbwertszeiten𝑇1/2bekannt sein.
 Dazu wird der Zeitgeber auf900sgesetzt und die angezeigte Zählrate notiert.
  Dann wird die erste Probe, ein116 49In-Präparat, in die Apparatur, zu
  sehen in Abbil-dung ??, eingesetzt. Der Zeitgeber wird auf ein
   Zeitintervall𝛥𝑡von240seingestellt. Es wird eine Stunde lang
   gemessen und die Zählraten werden jeweils notiert.
   Danach wird die Probe aus der Apparatur entfernt.
   Als letztes wird ein104Rhund104iRhGemisch untersucht.
   Die Probe wird in die Apparatur eingesetzt und das zu
    messende Zeitintervall wird auf20sgesetzt.
    Die Probe wird 12 Minuten lang beobachtet und die
     Zählraten werden notiert. Die verwendeten
     Zeitintervalle und Beobachtungszeiträume wurden
     in einer beiliegenden Tabelle vorgeschlagen und nicht selbst bestimmt.





\begin{table}
  \centering
  \caption{Messaufnahme von Silber (Ag).}
  \label{tab:1}
  \begin{tabular}{c c c c c c c c}
    \toprule
    $t \, /\, s$& $N_{\Delta t}$& $t \, /\, s$& $N_{\Delta t}$ & $t \, /\, s$& $N_{\Delta t}$ & $t \, /\, s$& $N_{\Delta t}$ \\
    \midrule
    10 & 387 & 130 & 25 & 250 & 10 & 370 & 8\\
    20 & 143 & 140 & 25 & 260 & 14 & 380 & 5\\
    30 & 113 & 150 & 30 & 270 & 20 & 390 & 11\\
    40 & 116 & 160 & 27 & 280 & 13 & 400 & 8\\
    50 & 82 & 170 & 13 & 290 & 10 & 410 & \cellcoller{grey}2\\
    60 & 58 & 180 & 27 & 300 & 30 & 420 & 11\\
    70 & 60 & 190 & 18 & 310 & 10 & -\\
    80 & 45 & 200 & 21 & 320 & 9 & -\\
    90 & 51 & 210 & 17 & 330 & 10 & -\\
    100 & 36 & 220 & 10 & 340 & 15 & -\\
    110 & 33 & 230 & 22 & 350 & 10 & -\\
    120 & 31 & 240 & 14 & 360 & 12 & -\\
    \bottomrule
  \end{tabular}
\end{table}

\begin{table}
  \centering
  \caption{Messaufnahme von Indium (In).}
  \label{tab:2}
  \begin{tabular}{c c c c}
    \toprule
    $t \, /\, s$& $N_{\Delta t}$& $t \, /\, s$& $N_{\Delta t}$ \\
    \midrule
    240  &  2930 & 2160 & 1773
    480  &  2710 & 2400 & 1710
    720  &  2604 & 2640 & 1710
    960  &  2258 & 2880 & 1543
    1200 &  2229 & 3120 & 1514
    1440 &  2117 & 3360 & 1446
    1680 &  1900 & 3600 & 1436
    1920 &  1923 & - & -
    \bottomrule
  \end{tabular}
\end{table}
