\section{Diskussion}
Bei der Bestimmung der Halbwertszeit von Indium sind die Messwerte alle ziemlich
gut auf einer Geraden und da der Fehler durch den Nulleffekt sehr klein ist, lässt
das auf eine gute Messreihe schließen. Die ermittelte Halbwertzeit war
$T_{In} = \SI{3160(140)}{\second}$ und der Theoriewert beträgt $T_\text{Theo, In}
= \SI{3257.4}{\second}$ \cite{2}. Das ergibt eine relative Abweichung von $2,99 \%$,
was im Toleranzbereich liegt.

Für Silber ist die Bestimmung der Halbwertszeit schwieriger, da dort zwei Zerfälle
gleichzeitig stattfinden. Außerdem wird die Zählrate am Ende der Messreihe sehr
klein, welche das Zählrohr nicht genau genug aufnehmen kann. Deshalb musste auch ein
Wert bei der Auswertung vernachlässigt werden. Da die Zeit $t^*$ willkürlich festgelegt
werden muss, kann dies auch zu Fehlern geführt haben.
Die bestimmten Halbwertzeiten waren für den kurzlebigen Zerfall $T_\text{kurz} = \SI{21.4(17)}{\second}$
und für den langlebigen Zerfall $T_\text{lang} = \SI{150(80)}{\second}$.
Werden diese Werte nun mit den Theoriewerten von $^{108}\text{Ag}$ und $^{110}\text{Ag}$ verglichen,
folgt dass $^{108}\text{Ag}$ für den langlebigen Zerfall verantwortlich ist und $^{110}\text{Ag}$
für den kurzlebigen.
Von $^{108}\text{Ag}$ ist der Theoriewert $T_\text{lang,theo} = \SI{142.2}{\second}$ \cite{2} was eine
Abweichung von $5,49 \%$ ergibt.
Für $^{110}\text{Ag}$ beträgt der Theoriewert $T_\text{kurz,theo} = \SI{24.6}{\second}$ \cite{2}.
Dabei ist die Abweichung somit $13,01 \%$.
Diese Abweichungen liegen beide im Toleranzbereich, was trotz der Fehlerquellen auf
eine gute Messung schließen lässt.
