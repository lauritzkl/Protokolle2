\section{Zielsetzung}
Es sollen die Halbwertzeiten von verschiedenen radioaktiven Isotopen bestimmt werden.
\section{Theorie}
Stabilen Atomkern sind sie, wenn die das Verhältnis zwischen Neutronen- und Protonanzahl innerhalb einer engen Grenzen
liegen.
Je nach Nuklid ist die Neutronenanzahl bis zu 50 \% höher im Kern vorhanden als Protoenen.
Aus diesem Zahlenverhältnis, außerhalb der Stabilitätsgrenze, zefällt sich mit unterschiedlicher Wahrscheinlichkeit
das Nuklid solange bis sie wieder stabil ist.
Die Zerfallswahrscheinlichkeit wird auch als Halbwertzeit $T_(\frac{1}{2})$ genannt und beschreibt
die Zeitspanne in welcher Größe und Aktiviät das Startnuklid um die Hälfte gesunken ist.
Die Halbwertzeit $T_(\frac{1}{2})$ variiert sich von Sekunden bis zu Jahren je nach Nuklid.
Zu Herstellung von radioaktiven Isotopen werden stabile Atomkerne mit Neutronen beschossen.
Dringt ein Neutron in ein Atomkern ein, so wird es Zwischenkern oder Compoundkern genannt.
Beim eindringen verteilt sich die Energie auf die große Anzahl der Nukleonen im Kern und ist nun instabil.
Da die kinetische Energie des Neutrons ein Nukleon abzustoßen zu gering ist, geht der angeregte Zwischenkern durch
Emission eines $\gamma \m$ Quant wieder in den Grundzustand zurück.
Der neu enstandene Kern ist langlebiger als der Zwischenkern und wandelt sich unter $\beta \m$ Strahlung in einen stabilen
Kern um.
\begin{equation*}
  gelb 1
\end{equation*}
Das sogenannte Wirkunsquerschnitt $\roh$ beschreibt die Wahrscheinlichkeit für die Reaktion eines Neutrons mit einem stabilen Kern.
Es beschreibt eine Fläche, die so groß ist, dass die treffenden Neutron vom Kern eingefangen werden müsste.
Dabei ist sie stark von der Geschwindigkeit $v$ und somit auch ihrer kinetische Energie des Neutrons abhängig. 
