\section{Diskussion}
Es werden die gemessenen Werte gegenüber dem Literaturwert gestellt. Sie sind in der
Tabelle \ref{tab:4} zu sehen.
\begin{table}[H]
  \centering
  \caption{Ergebnisse.}
  \label{tab:4}
  \begin{tabular}{c c c c}
    \toprule
     & $\text{gemssener Wert}\,/ \, \si{\milli\metre}$  & $\text{Literaturwert} \,/ \, \si{\milli\metre}$ & $\text{Abweichung}\, / \,\%$ \\
    \midrule
    $\text{Einfach Spalt 1}$ & $\num{0.1494(0019)}$ & 0,15 & 0,4\\
    $\text{Einfach Spalt 2}$ & $\num{0.07859(65)}$ & 0,075 & 4,79\\
    $\textbf{Doppelspalt}$   & -  & - &-\\
    $\text{Spaltbreite}$     & $\num{0.1633(0042)}$& 0,15 & 8,86\\
    $\text{Gitterkonstante}$ & $\num{0.509(3)}$ & 0,5 & 0,02\\
    \bottomrule
  \end{tabular}
\end{table}
Da aus der Theorie hervor geht, dass Licht gebeugt wird, wenn sie durch eine Öffnung hindurchtritt, die kleiner als der
Strahlendurchmesser ist. Dies bedeutet dass, für
die Wellenlänge $\lambda= \SI{635}{\nano\metre}$ die Spaltenbreite von $\SI{0.15}{\milli\metre}$ eine optimale
Breite ist. Dies widerspricht sich aber bei der Messung am Doppelspalt.
Somit kann die Aussage gemacht werden, dass die erste sowie letze Messung nicht präzise war und
die Messung am Spalt 2 die einzig aussagekräftige Messung ist. Da diese Messung unter der Toleranzgrene von $5 \%$
liegt, war die Messungen im ganzen gut.
