\section{Diskussion}
Es werden die gemessenen Werte gegenüber dem Literaturwert gestellt. Sie sind in der
Tabelle \ref{tab:4} zu sehen.
\begin{table}[H]
  \centering
  \caption{Ergebnisse.}
  \label{tab:4}
  \begin{tabular}{c c c c}
    \toprule
     & $\text{gemssener Wert}\,/ \, \si{\milli\metre}$  & $\text{Literaturwert} \,/ \, \si{\milli\metre}$ & $\text{Abweichung}\, / \,\%$ \\
    \midrule
    $\text{Einfach Spalt 1}$ & $\num{0.1481(0012)}$ & 0,15 & 1,27\\
    $\text{Einfach Spalt 2}$ & $\num{0.07859(65)}$ & 0,075 & 4,79\\
    $\textbf{Doppelspalt}$   & -  & - &-\\
    $\text{Spaltbreite}$     & $\num{0.1633(0042)}$& 0,15 & 8,86\\
    $\text{Gitterkonstante}$ & $\num{0.509(3)}$ & 0,5 & 0,02\\
    \bottomrule
  \end{tabular}
\end{table}

Zunächst fällt auf, dass einige Messwerte kleiner sind als $I_{du}$ was bedeutet, dass
$I_{du}$ nicht korrekt gemessen wurde und somit nicht in der Auswertung berücksichtig
werden konnte. Allerdings hätte $I_{du}$ nur für eine Verschiebung der Beugungsfiguren
nach unten gesorgt, also sind die bestimmten Spaltbreiten nicht beeinflusst.

Bei den Einzelspalten liegen die Abweichung im Toleranzbereich was auf eine gute Messung
schließen lässt.
Bei dem Doppelspalt hingegen ist die Abweichung der Spaltbreite am höhsten. Das könnte
dadran liegen, dass nicht genügend Messwerte aufgenommen wurden. Die Gitterkonstante
hingegen wurde sehr genau bestimmt.
