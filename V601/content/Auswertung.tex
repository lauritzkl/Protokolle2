\section{Auswertung}

Die mittlere freie Weglänge $\overline{w}$ und den Dampfdruck $p_\text{sätt}$, der bei den verschiedenen
Messreihen vorlag lässt sich mithilfe der Gleichung \ref{eq:6} bestimmen. Die Daten
sind in Tabelle \ref{tab:1} dargestellt.

\begin{table}
  \centering
  \caption{Die mittleren freien Weglängen und die Dampfdrücke von Hg bei verschiedenen
  Temperaturen.}
  \label{tab:1}
  \begin{tabular}{c c c}
    \toprule
    $T \, / \, K$ & $\overline{w} \, / \, cm$ & $p_\text{sätt} \, / \, mbar$ \\
    \midrule
    298,55 & $\num{5.30e-1}$ & $\num{5.47e-3}$ \\
    377,15 & $\num{4.36e-3}$ & $\num{6.60e-1}$ \\
    418,65 & $\num{7.16e-4}$ & $\num{4.05}$    \\
    465,15 & $\num{1.37e-4}$ & $\num{20.92}$ \\
    \bottomrule
  \end{tabular}
\end{table}

\subsection{Differentielle Energieverteilung}

Nun soll aus den in den Abbildungen !!!\ref{abb:4} und !!!\ref{abb:4} gemessenen Kurven
die differentielle Energieverteilung bestimmt werden. Dazu wird die Steigung
an verschiedenen Stellen gemessen und gegen die Bremsspannung $U_a$ aufgetragen.
Die gemessenen Steigungen von der ersten Messreihe, die bei $T = \SI{25.4}{\celsius}$
durchgeführt wurde, sind in Tabelle \ref{tab:2} dargestellt.

\begin{table}[H]
  \centering
  \caption{Darstellung der Steigungen a in Abhängigkeit von der Bremsspannung $U_a$
  der ersten Messreihe.}
  \label{tab:2}
  \begin{tabular}{c c}
    \toprule
    $U_a \, / \, V$ & $a$ \\
    \midrule
    0,40 & 0 \\
    0,80 & 0 \\
    1,20 & 0 \\
    1,60 & 0 \\
    2,00 & 0 \\
    2,60 & 0,033  \\
    4,00 & 0,025  \\
    5,20 & 0,050  \\
    5,80 & 0,100  \\
    6,24 & 0,083  \\
    6,72 & 0,083  \\
    7,12 & 0,083  \\
    7,68 & 0,110  \\
    8,00 & 0,125  \\
    8,22 & 0,200  \\
    8,42 & 0,200  \\
    8,62 & 0,200  \\
    8,82 & 0,200  \\
    9,00 & 0,250  \\
    9,16 & 0,250  \\
    9,38 & 0,286  \\
    9,64 & 0,330   \\
    9,84 & 0,500    \\
    \bottomrule
  \end{tabular}
\end{table}

Mit diesen Messwerten ergibt sich nun der Graph, welcher in Abbildung \ref{abb:3}
dargestellt ist.

\begin{figure}[H]
  \centering
  \includegraphics[width=\textwidth]{plot1.pdf}
  \caption{Graphische Darstellung der differentiellen Energieverteilung bei $T = \SI{25.4}{\celsius}$.}
  \label{abb:3}
\end{figure}

Da die Elektronen mit einer Spanunng von \SI{11}{\volt} beschleunigt werden lässt
sich aus dem Graphen das Kontaktpotential K der Apparatur berechnen, welches in diesem
Fall $ K = \SI{1.16}{\volt}$ ist.

Die zweite Messreiche wird bei $ T = \SI{145.5}{\celsius}$ durchgeführt.
Aufgrund der höheren Temperatur sind die bestimmten Werte auch anders als bei der
ersten Messreihe. Die berechneten Steigungen sind in Tabelle \ref{tab:3} dargestellt
und auch in diesem Fall werden die differentiellen Energieverteilungen gegen
die Bremsspannung in Abbildung \ref{abb:4} dargestellt.

\begin{table}[H]
  \centering
  \caption{Darstellung der Steigungen a in Abhängigkeit von der Bremsspannung $U_a$
  der zweiten Messreihe.}
  \label{tab:3}
  \begin{tabular}{c c}
    \toprule
    $U_a \, / \, V$ & $a$ \\
    \midrule
    0,08 & 0,750  \\
    0,28 & 0,830  \\
    0,54 & 0,714  \\
    0,86 & 0,667  \\
    1,24 & 0,750  \\
    1,66 & 0,462  \\
    2,12 & 0,300  \\
    2,60 & 0,100  \\
    3,00 & 0,100  \\
    3,40 & 0,200  \\
    3,80 & 0,200  \\
    4,20 & 0,200  \\
    4,60 & 0,200  \\
    5,00 & 0,300  \\
    5,40 & 0,100  \\
    5,80 & 0,100  \\
    6,20 & 0,100  \\
    6,60 & 0,100  \\
    7,20 & 0,050  \\
    8,00 & 0  \\
    8,40 & 0  \\
    8,60 & 0  \\
    \bottomrule
  \end{tabular}
\end{table}

\begin{figure}[H]
  \includegraphics[width=\textwidth]{plot2.pdf}
  \caption{Graphische Darstellung der differentiellen Energieverteilung bei $T = \SI{145.5}{\celsius}$.}
  \label{abb:4}
\end{figure}

Die Form der Messwerte der zweiten Messreihe unterscheidet sich so stark von der
ersten, weil bei der höheren Temperatur die mittlere freie Weglänge viel geringer ist
und die Elektronen mit einer höheren Wahrscheinlichkeit mit den Hg-Atomen wechselwirken.

\subsection{Franck-Hertz-Kurve}
Aus der in Abbildung \ref{abb:} gemessenen Franck-Hertz-Kurve wird nun die Anregungsenergie
des Hg-Atoms bestimmt indem die Abstände der Maxima gemessen werden. Die Ergebnisse
aus der Messung sind in Tabelle \ref{tab:4} gezeigt. Dabei gibt k an welches Maximum verwendet
wurde und $\Delta U = U_{k+1}-U_k$.

\begin{table}[H]
  \centering
  \caption{Abstände der Maxima aus der Franck-Hertz-Kurve.}
  \label{tab:4}
  \begin{tabular}{c c}
    \toprule
    $k$ & $\Delta U$ \\
    \midrule
    1 & 4,60 \\
    2 & 4,53 \\
    3 & 4,77 \\
    4 & 4,10 \\
    5 & 4,60 \\
    6 & 5,00 \\
    7 & 5,62 \\
    \bottomrule
  \end{tabular}
\end{table}

Aus den gemessenen Werten lässt sich nun der Mittelwert und die Standardabweichung
mithilfe der folgenden Formeln bestimmen.

\begin{gather*}
  \overline{U} = \frac{1}{N} \sum^N_{i=1} U_i \\
  \Delta \overline{U} = \frac{1}{\sqrt{N}} \sqrt{\frac{1}{N-1} \sum^N_{i=1}(U_i-\overline{U})^2}
\end{gather*}

Die Anregungsenergie lasst sich aus dem Mittelwert bestimmen indem der gemessene Abstand
mit der Elementarladung multipliziert wird.

\begin{equation*}
  \overline{U} = \SI{4.746(437)}{\electronvolt}.
\end{equation*}

Mithilfe der Gleichung \ref{eq:3} und der Relation $\lambda = \frac{c}{\nu}$ kann
nun die Wellenlänge der emittierten elektromagnetischen Strahlung bestimmt werden
mit der Formel

\begin{equation*}
  \lambda = h \frac{c}{E}
\end{equation*}

Der Fehler der Wellenlänge wird mit der Gauß´schen Fehlerfortpflanzung berechnet:

\begin{equation*}
  \Delta \lambda = \sqrt{\left(-\frac{hc}{E^2}\right)^2 \cdot \Delta E^2}
\end{equation*}

Damit ergibt sich für die Wellenlänge:

\begin{equation*}
  \SI{261(24)}{\nano\meter}
\end{equation*}

\subsection{Ionisierungsspannung}

In diesem Teil wird die Ionisationsspannung von Hg bestimmt. Die gemessenen Werte
von diesem Aufgabenteil sind in Abbildung \ref{abb:} dargestellt. Diese Kurve unterscheidet sich
von der erwarteten Theoriekurve. Auf die möglichen Gründe wird in der Diskussion eingegangen.
Für die bestimmung der Ionisationsspannung wird das erste Maximum der Kurve verwendet.
Das Maximum liegt bei einer Spannung von \SI{31.92}{\volt}. Von dieser Spannung muss nun noch
das Kontaktpotential, was zuvor bestimmt wurde, abgezogen werden um die Ionisationsspannung
zu erhalten.

\begin{equation*}
  U_\text{ion} = \SI{30.76}{\volt}
\end{equation*}
