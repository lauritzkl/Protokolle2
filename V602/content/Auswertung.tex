\section{Auswertung}
\subsection{Vorbereitung}
Die Literaturwerte der charakteristischen Röntgenstrahlung von Kupfer und die zugehörigen
Glanzwinkel $\theta$ sind \cite{2}:
\begin{itemize}
  \item $\text{Cu-}K_\alpha \text{-Linie} = 8,048 \,\text{keV} , \, \, \theta_\alpha = 22.49^\circ$
  \item $\text{Cu-}K_\beta \text{-Linie} = 8,905 \,\text{keV} , \, \, \theta_\beta = 20,22^\circ$

\end{itemize}

Des Weiteren sind die Literaturwerte der K-Kante von verschiedenen Materialien und die zugehörigen
Braggwinkel $\theta$ sowie die Abschirmkonstante $\sigma_k$ in der Tabelle (\ref{tab:1}) aufgelistet.
\begin{table}
  \centering
  \caption{Eigenschaftendarstellung der Materialien \cite{3}.}
  \label{tab:1}
  \begin{tabular}{c c c c c}
    \toprule
     & $Z$ & $E_{lit}\,/ \, \text{keV}$ &$\theta_{lit}\,/\,\circ$ & $\sigma_k$ \\
    \midrule
    Zn & 30 & 9,65  &  18.6 & 3.56 \\
    Ge & 32 & 11,1  &  16,1 & 3,43 \\
    Br & 35 & 13,47 &  13,2 & 3,53 \\
    Rb & 37 & 15,2  &  11,6 & 3,57 \\
    Sr & 38 & 16,1  &  11   & 3,59 \\
    Zr & 40 & 17,99 &  9,85 & 3,63 \\
    Nb & 41 & 18,98 &  9,33 & 3,64 \\
    \bottomrule
  \end{tabular}
\end{table}

\subsection{Überprüfung der Bragg Bedingung}
Es werden die Voreinstellungen die in der Durchführung angesprochen worden sind
durchgeführt und die Ergebnisse in der Tabelle (\ref{tab:2})
dargestellt als auch in der Abbildung (\ref{Bild:1})

\begin{table}[H]
  \centering
  \caption{Darstellung der Messreihe bei einer Zählrohwinkelrate bei 35 kV.}
  \label{tab:2}
  \begin{tabular}{c c c c}
  \toprule
  $2\cdot\theta \, / \, \circ$&	$I \, / \, \text{Imp/s}$ &$2\cdot\theta \, / \, \circ$&	$I \, / \, \text{imp/s}$ \\
  \midrule
  26,0&	31,0 &  28,1 &	156,0 \\
  26,1&	33,0 &  28,2 &  152,0 \\
  26,2&	42,0 &  28,3 &  150,0 \\
  26,3&	47,0 &  28,4 &  152,0 \\
  26,4&	48,0 &  28,5 &  138,0 \\
  26,5&	51,0 &  28,6 &  130,0 \\
  26,6&	58,0 &  28,7 &  126,0 \\
  26,7&	63,0 &  28,8 &  115,0 \\
  26,8&	60,0 &  28,9 &   97,0 \\
  26,9&	87,0 &  29,0 &   92,0 \\
  27,0&	84,0 &  29,1 &   88,0 \\
  27,1&	93,0 &  29,2 &   73,0 \\
  27,2&	100,0&  29,3 &   58,0 \\
  27,3&	112,0&  29,4 &   53,0 \\
  27,4&	113,0&  29,5 &   42,0 \\
  27,5&	116,0&  29,6 &   38,0 \\
  27,6&	117,0&  29,7 &   30,0 \\
  27,7&	132,0&  29,8 &   26,0 \\
  27,8&	136,0&  29,9 &   33,0 \\
  27,9&	137,0&  30,0 &   35,0 \\
  28,0&	147,0&    -  &     -  \\
 \bottomrule
\end{tabular}
\end{table}
Der Sollwinkel liegt bei $28^\circ$. Die aufgezeichnete Messreihe zeigen ein Maximum bei $28,1^\circ$.
Die Abweichung beträgt $0,1^\circ$ ($\approx 0,36\%$) und liegt im Toleranzbereich.

\subsection{Emissionsspektrum der Cu-Röntgenröhre}
Zur Aufnahme für das Emissionsspektrum wird der Koppelmodus 2:1 aktiviert. Im Anhang unter der Abbildung (\ref{Bild:2})
sind die Messergebnisse graphisch dargestellt. Zudem zeigen sie einen Bremsberg als auch die
$K_\beta \text{-Linie}$ und die $K_\alpha \text{-Linie}$ an.

Aus dem Grenzwinkel $\theta$ wird die minimale Wellenlänge bzw. die maximale Energie bestimmt.
Mit der Formel (\ref{eq:5}) und einer Gitterkonstante von $d=201,4 pm$  ist die minimale Wellenlänge bei
\begin{equation*}
  \lambda_{\text{min}} = \num{3.811e-11} \text{m}
\end{equation*}
Mit der Formel (\ref{eq:1}) ist die dazugehörige maximale Energie bei
\begin{equation*}
  E_{\text{max}} = \num{5.212e-15} \text{J} = 32,53 \, \text{keV}
\end{equation*}
Der erwartete Wert soll bei $E = 35 \,\text{keV}$ liegen.
Damit ist die Abweichung bei $7,06 \%$.\\

Mit der Full Width at Half Maximum Methode wird die Energieauflösung der $K_\alpha \text{- und der} \, K_\beta \text{-Linien}$ in Abbildung (\ref{Bild:2}) bestimmt.\\
Für $K_\beta$ folgt:
\begin{itemize}
  \item $\theta_1 = 39,52^\circ \Rightarrow E_1 = 9,1 \, \text{keV}$
  \item $\theta_2 = 40,95^\circ \Rightarrow E_2 = 8,8 \, \text{keV}$
\end{itemize}
Für die Umrechnung vom Braggwinkel zur Energie werden die Formelen (\ref{eq:5}) und (\ref{eq:1}) benutzt.
Die Energiedifferenz $\Delta E = E_1 - E_2 = 0,3 \, \text{keV}$ ist die gesuchte Energieauflösung.\\

Für $K_\alpha$ folgt:
\begin{itemize}
  \item $\theta_1 = 44,05^\circ \Rightarrow E_1 = 8,2 \, \text{keV}$
  \item $\theta_2 = 45,24^\circ \Rightarrow E_2 = 8 \, \text{keV}$
\end{itemize}
Die Energiedifferenz hier ist bei $\Delta E = 0,2 \,\text{keV}$ und die gesuchte Energieauflösung.

Anschließend werden die Abschirmkonstanten für $K_\alpha$ und für $K_\beta$ bestimmt.
Zunächst werden die Energie der $K_\beta \, \text{-und}\, K_\alpha$ -Peaks von Abbildung (\ref{Bild:2}) bestimmt.
Die folgende Umrechnung ist die selbe wie oben.
\begin{itemize}
  \item $\theta_\alpha = 44,52^\circ \Rightarrow E_\alpha = 8,13 \,\text{keV}$
  \item $\theta_\beta = 40,24^\circ \Rightarrow E_\beta = 8,95 \,\text{keV}$
\end{itemize}
Die Energiedifferenz der beiden Absorptionsenergien ist $\Delta E = 0,82 \, \text{keV}$
Durch Einsetzen in die Gleichung (\ref{eq:2}) folgt
\begin{equation*}
  \Delta E = R_\infty \cdot (z_{cu} - \sigma)^2 \cdot [(\frac{1}{1^2}-\frac{1}{2^2})-(\frac{1}{1^2}-\frac{1}{3^2})]
\end{equation*}
Durch umstellen und einsetzen der Werte ergibt sich für $\sigma$:
\begin{equation*}
  \sigma = z_{cu} - \sqrt{\frac{\Delta E \cdot 36}{R_\infty \cdot 5}} = 9,17
\end{equation*}

\subsection{Absorptionsspektrum}
Für diesen Versuchsteil werden folgende Absorber benutzt: Brom (Br), Strontium (Sr) und Zirkonium (Zr).
Die jeweiligen Ergebnisse zu den Messreihen sind unter Anhang graphisch dargestellt.
\subsubsection{Brom}
Aus der Absorptionskurve in Abbildung (\ref{Bild:3}) kann folgender Winkel abgelesen werden und mit der Formel (\ref{eq:5}) und (\ref{eq:1}) die
Absorptionsenergie bestimmt werden
\begin{itemize}
  \item $\theta_{Br} = 13,3^\circ \Rightarrow E_{Br} = 13,38 \, \text{keV}$
\end{itemize}
Nun wird nach der Formel (\ref{eq:2}) die Abschirmkonstante $\sigma_{br}$ bestimmt.
\begin{itemize}
  \item $\sigma_{Br} = Z_{Br} - \sqrt{\frac{E_{Br}}{R_\infty}} = 3,63$
\end{itemize}
\subsubsection{Strontium}
Aus der Absorptionskurve in Abbildung (\ref{Bild:4}) kann folgender Winkel abgelesen werden und mit der Formel (\ref{eq:5}) und (\ref{eq:1}) die
Absorptionsenergie bestimmt werden
\begin{itemize}
  \item $\theta_{Sr} = 11,1^\circ \Rightarrow E_{Sr} = 15,99 \,\text{keV}$
\end{itemize}
Nun wird nach der Formel (\ref{eq:2}) die Abschirmkonstante $\sigma_{Sr}$ bestimmt.
\begin{itemize}
  \item $\sigma_{Sr} = Z_{Sr} - \sqrt{\frac{E_{Sr}}{R_\infty}} = 3,71$
\end{itemize}
\subsubsection{Zirkonium}
Aus der Absorptionskurve in Abbildung (\ref{Bild:5}) kann folgender Winkel abgelesen werden und mit der Formel (\ref{eq:5}) und (\ref{eq:1}) die
Absorptionsenergie bestimmt werden
\begin{itemize}
  \item $\theta_{Zr} = 10,19^\circ \Rightarrow E_{Zr} = 17,4 \, \text{keV}$
\end{itemize}
Nun wird nach der Formel (\ref{eq:2}) die Abschirmkonstante $\sigma_{Zr}$ bestimmt.
\begin{itemize}
  \item $\sigma_{Zr} = Z_{Zr} - \sqrt{\frac{E_{Zr}}{R_\infty}} = 4,23$
\end{itemize}

Nach dem Moseleyschen Gesetz ist die Absorptionsenergie $E_n$ proportional zu Z$^2$ siehe Formel (\ref{eq:2}).
Zur Bestimmung der Rydbergenergie $R_\infty$ wird sie umgeformt zu:
\begin{equation*}
  \sqrt{E_n} = \sqrt{R_\infty} \cdot Z - \sqrt{R_\infty} \cdot \sigma
\end{equation*}
Zu sehen ist eine lineare Gleichung mit der $\sqrt{R_\infty}$ als Steigung m und $\sqrt{R_\infty} \cdot \sigma$ als
y-Achsenschnitt b.
In Abbildung (\ref{abb:3}) sind die Messergebnisse graphisch dargestellt und mit Python 3.6.4 wurde
die Regressionsgerade berechnet.
\begin{figure}[H]
  \centering
  \includegraphics[width=\textwidth]{plot1.pdf}
  \caption{Graphische Darstellung zur Bestimmung der Rydbergenergie.}
  \label{abb:3}
\end{figure}
\begin{itemize}
  \item $m =  3,275 \pm 0,236 \, \text{$\sqrt{eV}$}$
  \item $b =  1,308 \pm 8,907$
\end{itemize}
Somit ergbit sich für die gemessene Rydbergenergie $R_{\infty,\text{mes}} = 10,73 \, \text{eV}$.
Dies weicht sich von der bekannten Rydbergenergie $R_{\infty, \text{lit}} = 13,6 \, \text{eV}$
um $21,1 \%$ ab.


\subsubsection{Abschirmungszahl bei Mehrelektronenatomen}
In dieser Versuchsreihe wird der Absorber Wismut (Wi) untersucht.
Die aufgezeichneten Messdaten sind unter Anhang in der Abbildung(\ref{Bild:6}) graphisch dargestellt.
Es wird nun die Absorptionsenergie der beiden K-Kanten, wie oben schon angewendet, ausgerechnet.
\begin{itemize}
  \item $\theta_{1,Wi} = 11,5^\circ \Rightarrow E_{1,Wi} = 15,44 \,\text{keV}$
  \item $\theta_{2,Wi} = 13,38^\circ \Rightarrow E_{2,Wi} = 13,31 \,\text{keV}$
\end{itemize}
Die Energiedifferenz der beiden Absorptionsenergie ist $\Delta E = 2,13 \,\text{keV}$.
Nun wird die Abschirmkonstante $\sigma_L$ mit der Formel (\ref{eq:4}) bestimmt, dabei ist Z = 83
\begin{itemize}
  \item $\sigma_L = 4,85 $
\end{itemize}
