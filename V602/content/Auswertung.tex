\section{Auswertung}
\subsection{Vorbereitung}
Die Literaturwerte der charakteristischen Röntgenstrahlung von Kupfer und die zugehörigen
Glanzwinkel $\theta$ sind \cite{2}:
\begin{itemize}
  \item Cu-K_\alpha - Linie = 8,048 KeV
  \item \theta_\alpha = 22.49^\circ
  \item Cu-K_\beta -Linie = 8,905 KeV
  \item \theta_\beta = 20,22^\circ
\end{equation}

Des Weiteren sind die Literaturwerte der K-Kante von verschiedenen Materialien und die zugehörigen
Braggwinkel $\theta$ sowie die Abschirmkonstante $\roh_k$ in der Tabelle (\ref{tab:1}) aufgelistet.
\begin{table}
  \centering
  \caption{Eigenschaftendarstellung der Materialien \cite{3}.}
  \label{tab:1}
  \begin{tabular}{c c c c c}
    \toprule
     & $Z$ & $E_{lit}\,/\,KeV$ &$\theta_{lit}\,/\,\circ$ & $\roh_k$
    \midrule
    Zn & 30 & 9,65  &  18.6 & 3.56
    Ge & 32 & 11,1  &  16,1 & 3,43
    Br & 35 & 13,47 &  13,2 & 3,53
    Rb & 37 & 15,2  &  11,6 & 3,57
    Sr & 38 & 16,1  &  11   & 3,59
    Zr & 40 & 17,99 &  9,85 & 3,63
    Nb & 41 & 18,98 &  9,33 & 3,64
    \bottomrule
  \end{tabular}
\end{table}

\subsection{Überprüfung der Bragg Bedingung}
Es werden wie oben beschrieben die Voreinstellungen eingestellt und in der Tabelle (\ref{tab:2})
dargestellt.
\begin{table}
  \caption{Darstellung der Messreihe bei einer Zählrohwinkelrate bei 35kV.}
  \label{tab:2}
  \begin{tabular}{c c c c}
  \toprule
  $2*\theta \, / \, \circ$&	$I \, / \, Imp/s$ &$2*\theta \, / \, \circ$&	$I \, / \, imp/s$
  \midrule
  26,0&	31,0 &  28,1 &	156,0
  26,1&	33,0 &  28,2 &  152,0
  26,2&	42,0 &  28,3 &  150,0
  26,3&	47,0 &  28,4 &  152,0
  26,4&	48,0 &  28,5 &  138,0
  26,5&	51,0 &  28,6 &  130,0
  26,6&	58,0 &  28,7 &  126,0
  26,7&	63,0 &  28,8 &  115,0
  26,8&	60,0 &  28,9 &   97,0
  26,9&	87,0 &  29,0 &   92,0
  27,0&	84,0 &  29,1 &   88,0
  27,1&	93,0 &  29,2 &   73,0
  27,2&	100,0&  29,3 &   58,0
  27,3&	112,0&  29,4 &   53,0
  27,4&	113,0&  29,5 &   42,0
  27,5&	116,0&  29,6 &   38,0
  27,6&	117,0&  29,7 &   30,0
  27,7&	132,0&  29,8 &   26,0
  27,8&	136,0&  29,9 &   33,0
  27,9&	137,0&  30,0 &   35,0
  28,0&	147,0&
 \bot &tomrule
\end{table}
Der Sollwinkel liegt bei $28^\circ$. Die aufgezeichnete Messreihe zeigen ein Maximum bei $28,1^\circ$.
Die Abweichung beträgt $0,1^\circ$ und liegt im Toleranzbereich.

\subsection{Emissionsspektrum der Cu-Röntgenröhre}
Zur Aufnahme für das Emissionsspektrum wird der Koppelmodus 2:1 aktiviert. Im Anhang unter der Abbildung (Bild)
sind die Messergebnise graphisch dargestellt. Zudem zeigen sie einen Bremsberg als auch die
$K_\beta -Linie$ und die $K_\alpha -Linie$ an.

Aus dem Grenzwinkel $\theta$ wird die minimale Wellenlänge bzw. die maximale Energie bestimmt.
Mit der Formel (\ref{eq:3}) und einer Gitterkonstante $d=204,1 pm$  ist die minimale Wellenlänge bei
\begin{equation*}
  \lambda_{min} = \num{3.811e-11} \text{m}
\end{equation*}
Mit der Formel (\ref{eq:1}) ist die dazugehörige maximale Energie
\begin{equation*}
  E_{max} = \num{5.212e-15} \text{J} = 32,53 \text{KeV}
\end{equation*}
Der erwartete Wert soll bei $E = 35 \text{KeV}$ liegen.
Damit ist die Abweichung bei $7,06 \%$.

Mit der Full Width at Half Maximum Methode wird die Energieauflösung der $K_\alpha - und der K_\beta Linien$ in Abbildung (Bild) bestimmt.
Für $K_\beta$ folgt
\begin{itemize}
  \item \theta_1 = 39,52^\circ \rightarrow E_1 = 9,1 \text{KeV}
  \item \theta_2 = 40,95^\circ \rightarrow E_2 = 8,8 \text{KeV}
\end{itemize}
Für die Umrechnung vom Braggwinkel zur Energie werden die Formelen (\ref{eq:5}) und (\ref{eq:1}) benutzt.
Die Energiedifferenz $\Delta E = E_1 - E_2 = 0,3 \text{KeV}$ ist die gesuchte Energieauflösung.

Für $K_\alpha$ folgt
\begin{itemize}
  \item \theta_1 = 44,05^\circ \rightarrow E_1 = 8,2 \, \text{KeV}
  \item \theta_2 = 45,24^\circ \rightarrow E_2 = 8 \, \text{KeV}
\end{itemize}
Wie oben ist die Energiedifferenz $\Delta E = 0,2 \text{KeV}$ die gesuchte Energieauflösung.

Anschließend werden die Abschirmkonstanten für $K_alpha$ und für $K_beta$ bestimmt.
Zunächst werden die Energie der $K_\beta und K_\alpha$ Peaks von Abbildung (Bild) bestimmt.
Die folgende Umrechnung ist die selbe wie oben.
\begin{itemize}
  \item \theta_\alpha = 44,52^\circ \rightarrow E_\alpha = 8,13 \text{KeV}
  \item \theta_\beta = 40,24^\circ \rightarrow E_\beta = 8,95 \text{KeV}
\end{itemize}
Die Energiedifferenz der beiden Absorptionsenergien ist $\Delta E = 0,82 \text{KeV}$
Durch Einsetzen in die Gleichung (\ref{eq:2}) folgt
\begin{equation*}
  \Delta E = R_\infty \cdot (z_{cu} - \roh)^2 \cdot [(\frac{1}{1^2}-\frac{1}{2^2})-(\frac{1}{1^2}(\frac{1}{3^2}))]
\end{equation*}
Durch umformen der Gleichung nach Abschirmkonstante $\roh$ ergibt sich:
\begin{equation*}
  \roh = z_{cu} - \sqrt{\frac{\Delta E \cdot 36}{R_\infty \cdot 5}} = 9,17
\end{equation*}

\subsection{Absorptionsspektrum}
Für diesen Versuchsteil werden folgende Absorber benutzt: Brom (Br), Strontium (Sr) und Zirkon (Zr).
Die jeweiligen Ergebnisse zu den Messreihen sind unter Anhang graphisch dargestellt.
\subsubsection{Brom}
Aus der Absorptionskurve in Abbildung (Brom) kann folgender Winkel abgelesen werden und mit der Formel (\ref{eq:5}) und (\ref{eq:1}) die
Absorptionsenergie bestimmt werden
\begin{itemize}
  \item \theta_{Br} = 13,3^\circ \rightarrow E_{Br} = 13,38 \text{KeV}
\end{itemize}
Nun wird nach der Formel (\ref{eq:2}) die Abschirmkonstante $\roh_{br}$ bestimmt.
\begin{itemize}
  \item \roh_{Br} = Z_{Br} - \sqrt{\frac{E_{Br}}{R_\infty}} = 3,63
\end{itemize}
\subsubsection{Strontium}
Aus der Absorptionskurve in Abbildung (Strontium) kann folgender Winkel abgelesen werden und mit der Formel (\ref{eq:5}) und (\ref{eq:1}) die
Absorptionsenergie bestimmt werden
\begin{itemize}
  \item \theta_{Sr} = 11,1^\circ \rightarrow E_{Sr} = 15,99 \text{KeV}
\end{itemize}
Nun wird nach der Formel (\ref{eq:2}) die Abschirmkonstante $\roh_{br}$ bestimmt.
\begin{itemize}
  \item \roh_{Sr} = Z_{Sr} - \sqrt{\frac{E_{Sr}}{R_\infty}} = 3,71
\end{itemize}
\subsubsection{Zirkon}
Aus der Absorptionskurve in Abbildung (Zirkon) kann folgender Winkel abgelesen werden und mit der Formel (\ref{eq:5}) und (\ref{eq:1}) die
Absorptionsenergie bestimmt werden
\begin{itemize}
  \item \theta_{Zr} = 10,19^\circ \rightarrow E_{Zr} = 17,4 \text{KeV}
\end{itemize}
Nun wird nach der Formel (\ref{eq:2}) die Abschirmkonstante $\roh_{br}$ bestimmt.
\begin{itemize}
  \item \roh_{Sr} = Z_{Zr} - \sqrt{\frac{E_{Zr}}{R_\infty}} = 4,23
\end{itemize}
\subsubsection{Abschirmungszahl bei Mehrelektronenatomen}
In dieser Versuchsreihe wird der Absorber Wismut (Wi) untersucht.
Die aufgezeichneten Messdaten sind unter Anhang in der Abbildung(Wismut) graphisch dargestellt.
Es wird nun die Absorptionsenergie der beiden K-Kanten, wie oben schon angewendet, ausgerechnet.
\begin{itemize}
  \item \theta_{1,Wi} = 11,5^\circ \rightarrow E_{1,Wi} = 15,44 \text{KeV}
  \item \theta_{2,Wi} = 26,76^\circ \rightarrow E_{2,Wi} = 6,84 \text{KeV}
\end{itemize}
Die Energiedifferenz der beiden Absorptionsenergie ist $\Delta E = 8,6 \text{KeV}$.
Nun wird die Abschirmkonstante $\roh_L$ mit der Formel (\ref{eq:4}) bestimmt, dabei Z = 83
\begin{itemize}
  \item \roh_L = -21,09
\end{itemize}
