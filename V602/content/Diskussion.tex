\section{Diskussion}
In der Tabelle (\ref{tab:3}) sind die Abweichungen der Absorptionsenergien aufgelistet.
\begin{table}
  \centering
  \caption{Vergleich von gemessenen Werten mit Literaturwerten.}
  \label{tab:3}
  \begin{tabular}{c c c c}
    \toprule
     & $E_{mes}\,/ \, \text{keV}$  & $E_{lit}\,/ \, \text{keV}$ & $Abweichung \,/\,\%$ \\
    \midrule
    Br & 13,47 & 13,38 & 0,67\\
    Sr & 16,10 & 15,99 & 0,68\\
    Zr & 17,99 & 17,40 & 3,28\\
    \bottomrule
  \end{tabular}
\end{table}
Ein möglicher Grund für die Abweichungen könnte durch nicht genau Winkeleinzeichung im Graphen entstehen.
Des Weiteren lag für die Bragg-Bedingung eine Abweichung von $0,36\%$ vor, was zu minimale Fehler bei der
Aufnahme führen kann.
Zu dem Absorber von Wismut finde ich keinen Fehler :)

\newpage
\section{Anhang}
\begin{figure}[p]
  \centering
  \includegraphics[width=\textwidth]{content/Ueberpruefung.jpg}
  \caption{Überprüfung der Bragg-Bedingung.}
  \label{Bild:1}
\end{figure}
\begin{figure}[p]
  \centering
  \includegraphics[width=\textwidth]{content/Peaks.jpg}
  \caption{Aufnahme zum Emissionsspektrum der Cu-Röntgenröhre.}
  \label{Bild:2}
\end{figure}
\begin{figure}[p]
  \centering
  \includegraphics[width=\textwidth]{content/Brom.jpg}
  \caption{Aufnahme mit dem Absorber Brom.}
  \label{Bild:3}
\end{figure}
\begin{figure}[p]
  \centering
  \includegraphics[width=\textwidth]{content/Strontium.jpg}
  \caption{Aufnahme mit dem Absorber Strontium.}
  \label{Bild:4}
\end{figure}
\begin{figure}[p]
  \centering
  \includegraphics[width=\textwidth]{content/Zirkon.jpg}
  \caption{Aufnahme mit dem Absorber Zirkon.}
  \label{Bild:5}
\end{figure}
\begin{figure}[p]
  \centering
  \includegraphics[width=\textwidth]{content/Bismut.jpg}
  \caption{Aufnahme mit dem Absorber Wismut.}
  \label{Bild:6}
\end{figure}
